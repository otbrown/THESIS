In the reference frame, the Hamiltonian for a driven nonlinear cavity is,
\begin{equation}
	\mathcal{H}_{\mathrm{ref}} = \omega\hat{a}^{\dagger}\hat{a} + \frac{U}{2}\hat{a}^{\dagger}\hat{a}^{\dagger}\hat{a}\hat{a} + \frac{1}{\sqrt{2}} \left(\Omega\mathrm{e}^{-i\omega_{L}t} + \tilde{\Omega}\mathrm{e}^{i\omega_{L}t}\right) \hat{a}^{\dagger}\hat{a}^{\dagger} + \frac{1}{\sqrt{2}} \left( \Omega^{*}\mathrm{e}^{i\omega_{L}t} + \tilde{\Omega}^{*}\mathrm{e}^{-i\omega_{L}t} \right)\hat{a}\hat{a},
	\label{eq:rot1}
\end{equation}
where \(\omega\) is the cavity frequency, \(U\) is the interaction strength, \(\Omega\) is the amplitude of the drive laser, \(\omega_{L}\) is the frequency of the drive laser, and \(\hbar = 1\). Our aim is to transform this in to a rotating frame such that the time dependence of the drive terms is eliminated from the Hamiltonian.

To achieve this the frame must rotate at the frequency of the driving laser. The appropriate transformation is,
\begin{equation}
	\mathcal{H}_{\mathrm{RF}} = \hat{V}^{\dagger}\mathcal{H}_{\mathrm{ref}}\hat{V} - \hat{A},
	\label{eq:rot2}
\end{equation}
where,
\begin{align}
	\hat{V} &= \mathrm{e}^{-i\hat{A}t}, \label{eq:rot3} \\
	\hat{A} &= \frac{\omega_{L}}{2} \hat{a}^{\dagger}\hat{a}, \label{eq:rot4}
\end{align}
where we note that \(\hat{A}\) is a diagonal operator, so its exponentiation is simple.
	