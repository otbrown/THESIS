In this section we consider variational searches using Matrix Product States. We will first discuss the theoretical underpinnings of such a technique -- what we vary, and what we search for. I will then describe details pertaining to my own implementations of these techniques. There is no suggestion that these implementation details represent best practise, or are in any sense \emph{the right way} to perform these calculations. They are merely how \emph{I} approached the problem. All these implementations were written in and for MATLAB \cite{MATLAB}, and can at time of writing be found in repositories hosted at Ref~\cite{otb:githome}.
 
 \subsection{Theory}
It is well known that one can use the Rayleigh-Ritz Variational Technique to find an approximation to the lowest eigenvalue and corresponding eigenfunction of a Hermitian operator. Given a set of variational parameters upon which the eigenfunctions depend, one can move always to a lower eigenvalue, by minimising over one parameter at a time \cite{ArfWeb_RRVT, Gasiorowicz_RVT}. Consequently, we can find an approximation to the ground state of a system by minimising the expression,
\begin{equation}
E = \frac{\langle \psi (\bar{x}^{*}) | \hat{H} | \psi (\bar{x}) \rangle}{\langle \psi (\bar{x}^{*}) | \psi (\bar{x}) \rangle},
\label{eq:vs1-1}
\end{equation}
where E is the energy of the system, \(\hat{H}\) is a Hamiltonian, \(\psi\) is an approximation to the ground state, and \( \mathbf{x} \) is some set of \emph{variational parameters}. Equally, we can find an approximation to the stationary state of an open quantum system by minimising the expression,
\begin{equation}
\frac{\mathrm{d}\rho}{\mathrm{d}t} = \langle \rho(\mathbf{x}^{*}) | \hat{\mathcal{L}^{\dagger}} \hat{\mathcal{L}} | \rho(\mathbf{x}) \rangle,
\label{eq:vs1-2}
\end{equation}
where \(\hat{\mathcal{L}}\) is a Liouvillian, \(\rho\) is an approximation to the stationary state, and \(\mathbf{x}\) is again some set of variational parameters. For the purposes of this theoretical discussion we will focus on the ground state case as the two cases are very similar, but there are additional complexities in the stationary state search. A visual representaion of the variational search procedure is provided in \cref{fig:vs1-1}.

\begin{figure}[ht!]
\centering
\includegraphics[width=0.8\linewidth]{\figpath/var_space}
\caption{A visual representation of a variational search using matrix product states. The purple background represents the total state space of the system, and the green oval is the part of that state space that can be represented by a matrix product state of some finite dimension. The orange star represents our desired solution state, and it is inaccessible to the matrix product state space. The black circle is the initial matrix product state, the black star is the nearest matrix product state approximation to the solution state, and the black squares are states through which the matrix product state transitions on its way to the solution state. The black dashed line represents a variational step -- optimisation over one or more of the variational parameters. The transitional states may or may not have some physical meaning in the context of the variational search depending on the specifics of the system being investigated. In general, however, if one wishes to know \emph{how} a system reaches the solution state a time evolution method should be used, not a variational search.}
\label{fig:vs1-1}
\end{figure}

When using Matrix Product States the set of variational parameters we employ are the individual site tensors. We shall discuss the search procedure as prescribed by Ulrich Schollw\"{o}ck's excellent review article \cite{Schollwoeck11}. I begin my explanation by assuming that we have already some initial matrix product state, \(\Psi_{\mathrm{Init}}\) which is normalised, and has dimensions \(N \times \chi_{\mathrm{max}} \times \chi_{\mathrm{max}} \times d\), where \(N\) is the number of sites in the system, \(\chi_{\mathrm{max}}\) is the maximal allowed matrix dimension, and \(d\) is the local state space dimension. Additionally I assume we have some observable we wish to minimise the expectation value of, with an operator \(\hat{O}\), which we may represent as a matrix product operator \(O^{[n]}\). First, we construct left and right `blocks' for each site in the system. The left block for some site \(n\) is a rank-3 tensor which contains the expectation of \(\hat{O}\) from the first site up to the site \(n-1\). The right block for some site \(n\) is a rank-3 tensor which contains the expectation of \(\hat{O}\) from last site through to the site \(n+1\). This is shown diagramatically in \cref{fig:vs1-2}.

\begin{figure}[ht!]
\centering
\includegraphics[width=0.8\linewidth]{\figpath/LR_blocks}
\caption{A tensor network diagram for a system which has been partially contracted in order to form left and right blocks, \(L^{[n]}\) and \(R^{[n]}\). The upper red dot here is a tensor for the site \(n\), \(A^{[n]}\), and the lower red dot is its conjugate, \(A^{\dagger [n]}\). The blue square is the mpo tensor \(O^{[n]}\) of some observable with an operator \(\hat{O}\). The black lines represent tensor indices which can be contracted over. If this contraction is completed it will be equivalent to a cointraction over the full system, and the result will be the expectation value \(\langle \Psi | \hat{O} | \Psi \rangle \).}
\label{fig:vs1-2}
\end{figure}

The first site left block tensor \(L^{[1]}\) is just the scalar \(1\), as there are obviously no sites before the first. The second left block tensor \(L^{[2]}\) is then found by performing the contraction procedure,
\begin{align}
L^{[2]}_{r^{\prime}, c, q} &= \sum_{\sigma^{\prime}, c^{\prime}} A^{\dagger [1] \sigma^{\prime}}_{r^{\prime}, c^{\prime}} \left( \sum_{\sigma, p} O^{[1]  \sigma, \sigma^{\prime}}_{p, q} \left( \sum_{r} L^{[1]}_{c^{\prime}, r, p} A^{[1] \sigma}_{r, c} \right) \right), \notag \\
&= \sum_{\sigma^{\prime}, c^{\prime}} A^{\dagger [1] \sigma^{\prime}}_{r^{\prime}, c^{\prime}} \left( \sum_{\sigma, p} O^{[1]  \sigma, \sigma^{\prime}}_{p, q} \left( \sum_{r} A^{[1] \sigma}_{r, c} \right) \right),
\label{eq:vs1-3}
\end{align}
where \(A^{[n]}\) is the matrix product state tensor for the site \(n\), \(\sigma\) indexes the local physical state, \(r\) and \(c\) (`row' and `column') index the local virtual dimensions, primed indices relate to the conjugate matrix product state tensor \(A^{\dagger [n]}\), and \(p\) and \(q\) index the virtual dimensions of the matrix product operator. The procedure continues from there, much as you might expect, by moving on to the third site and so on until the last site is reached. The general formula for \(L^{[n]}\) is,
\begin{equation} 
L^{[n]}_{r^{\prime}, c, q} = \sum_{\sigma^{\prime}, c^{\prime}} A^{\dagger [n-1] \sigma^{\prime}}_{r^{\prime}, c^{\prime}} \left( \sum_{\sigma, p} O^{[n-1]  \sigma, \sigma^{\prime}}_{p, q} \left( \sum_{r} L^{[n-1]}_{c^{\prime}, r, p} A^{[n-1] \sigma}_{r, c} \right) \right).
\label{eq:vs1-4}
\end{equation}

The procedure for forming the right block is naturally very similar, starting from the last site with \(R^{[N]} = 1\) and,
\begin{equation}
R^{[n]}_{c^{\prime}, r, p} = \sum_{\sigma^{\prime}, r^{\prime}} A^{\dagger [n+1] \sigma^{\prime}}_{r^{\prime}, c^{\prime}} \left( \sum_{\sigma, q} O^{[n+1] \sigma, \sigma^{\prime}}_{p, q} \left( \sum_{c} R^{[n+1]}_{r^{\prime}, c, q} A^{[n+1] \sigma}_{r, c} \right) \right).  
\label{eq:vs1-5}
\end{equation}
Once we have formed these left and right blocks at each site, we move on to the variational procedure proper. 

We will sweep backwards and forwards through the system, updating each site tensor to minimise the energy of the overall state. Referring back to \cref{eq:vs1-1} we can see that it can be minimised by being rephrased as an eigenvalue problem,
\begin{align}
\frac{\langle \psi(\mathbf{x}^{*}) | \hat{H} | \psi(\mathbf{x}) \rangle}{\langle \psi(\mathbf{x}^{*}) | \psi(\mathbf{x}) \rangle } &= E, \notag \\
\Rightarrow \langle \psi(\mathbf{x}^{*}) | \hat{H} | \psi(\mathbf{x}) \rangle &= E \langle \psi(\mathbf{x}^{*}) | \psi(\mathbf{x}) \rangle, \notag \\
\Rightarrow \frac{\mathrm{d}}{\mathrm{d}\langle \psi(\mathbf{x}^{*}) |} \left( \langle \psi(\mathbf{x}^{*}) | \hat{H} | \psi(\mathbf{x}) \rangle \right) &= \frac{\mathrm{d}}{\mathrm{d}\langle \psi(\mathbf{x}^{*}) |} \left(  E \langle \psi(\mathbf{x}^{*}) | \psi(\mathbf{x}) \rangle \right), \notag \\
\Rightarrow \hat{H} |\psi(\mathbf{x}) \rangle &= E | \psi(\mathbf{x}) \rangle,
\label{eq:vs1-6}
\end{align}
which of course is an expression of the time-independent Schr\"{o}dinger equation. If we could solve that for the many-body state \(| \psi (\mathbf{x}) \rangle\) then we would not need matrix product states at all. Unfortunately, we cannot, but what matrix product states allow us to do is to form an effective Hamiltonian for some particular \(| \psi(x) \rangle\), and instead solve the more limited eigenvalue problem,
\begin{equation}
\hat{H}_{\mathrm{eff}} |\psi(x) \rangle = E_{[x]} |\psi(x) \rangle,
\label{eq:vs1-7}
\end{equation}
from which we simply select \(|\psi(x) \rangle\) which corresponds to the lowest real value of \(E_{[x]}\). In our case \(|\psi(x) \rangle\) is \(A^{[n]}\), and \(\hat{H}_{\mathrm{eff}}\) is formed by the contraction of the \emph{environment} of \(A^{[n]}\) \cite{Orus14}. That is we calculate,
\begin{equation}
\hat{H}_{\mathrm{eff}}^{[n]} = \langle \psi(\tilde{\mathbf{x}}) | \hat{H} | \psi(\tilde{\mathbf{x}}) \rangle,
\label{eq:vs1-8}
\end{equation}
where \(|\psi(\tilde{\mathbf{x}}) \rangle \) is our matrix product state \emph{excluding the site \(n\)}. Such a contraction is shown diagramatically in \cref{fig:vs1-3}. Mathematically, the contraction is performed as,
\begin{equation}
\hat{H}^{[n]\, \mathrm{eff}}_{r,c,r^{\prime},c^{\prime},\sigma,\sigma^{\prime}} = \sum_{p,q} L^{[n]}_{r,r^{\prime},p} O^{[n] \sigma, \sigma^{\prime}}_{p,q} R^{[n]}_{c,c^{\prime},q},
\label{eq:vs1-9}
\end{equation}
which seems simple enough, and indeed would be except that we have an eigenvalue problem to solve. As such we require \(\hat{H}_{\mathrm{eff}}^{[n]}\) to be a matrix, not a rank-6 tensor. This can be accomplished by joining the indices corresponding to the matrix product state, and joining those of its conjugate to form a matrix \(\hat{H}^{[n]\, \mathrm{eff}}_{(\sigma ,r,c), (\sigma^{\prime},r^{\prime},c^{\prime})}\). Once this is achieved it is a simple matter of finding the eigenvector of \(\hat{H}^{[n]\, \mathrm{eff}}_{(\sigma ,r,c), (\sigma^{\prime},r^{\prime},c^{\prime})}\) with the lowest real eigenvalue. This eigenvector is the vectorised site tensor \(A^{[n]}_{\sigma, r, c}\), which we reshape to be \(A^{[n] \sigma}_{r,c}\) and update our matrix product state.

We do this on the first site in our system, and then re-normalise using an SVD or QR decomposition in order to make the site left-canonical. We then update our left block for the second site in the system, \(L^{[2]}\) using \cref{eq:vs1-4}. We are then ready to find an effective Hamiltonian for the second site and update it. This procedure repeats sweeping `right' through our system until we reach and update the \(N\)th site -- at this point we have updated every site in the system, but it is unlikely that our observable has converged after only one such sweep. The procedure for sweeping `left' through the system back to the first site is very similar, except when re-normalising we make our newly updated site right-canonical and then update the right block, \(R^{[n]}\). In this way we are always using the most up-to-date version of the system when we calculate the effective operator for a given site. The whole procedure repeats, sweeping left and right through the system until our chosen observable converges.

\begin{figure}[ht!]
\centering
\includegraphics[width=0.8\linewidth]{\figpath/effH_diagram}
\caption{A diagrammatic representation of the contraction that must be performed in order to find the effective operator on some site \(n\). As usual the red circles represent matrix product state tensors, the blue squares represent some matrix product operator, and black lines are indices. The lines which reach into the gap left by the missing site \(n\) are indices which are left free, and will become the indices of the effective operator. As such, it can be seen that the effective operator will be a rank-6 tensor.}
\label{fig:vs1-3}
\end{figure}

\FloatBarrier
 \subsection{Ground State Implementation}
 We will now discuss my specific implementation of the variational ground state search. The implementation is written for MATLAB \cite{MATLAB}, and at the time of writing is held in a git repository hosted at Ref~\cite{otb:gitVGSS}. In this section we will use the conventions that \(N\) is the number of sites in our system, and \(d\) is the dimension of the local state space, so the total state space of our system would be \(d^{N}\). \Cref{fig:vs2-1} shows the structure of the code diagramatically.
 
 \begin{figure}[ht!]
 \centering
 \includegraphics[width=0.8\linewidth]{\figpath/mpsvgss}
 \caption{A diagram showing the structure of the variational ground state search code. Each rectangle is a function which forms part of the code, while the arrows represent return values. The return value follows the direction of the arrow i.e. \lstinline$LCan$ is called from, and returns data to, \lstinline$Can$. It can be seen that only \lstinline$Ground$, \lstinline$Can$, and \lstinline$GrowBlock$ have dependencies. The matrix product state constructor functions \lstinline$CompMPS$, and \lstinline$ProdMPS$ are not shown as they are convenience functions which help to form part of the input to \lstinline$Ground$, but are not necessary to run the calculation. They both have only one dependency, on \lstinline$MPSNorm$. Note that the built-in functions of MATLAB \cite{MATLAB} are not considered, but almost all the functions written for the variational ground state search rely on them to some extent. They are assumed to operate properly.}
 \label{fig:vs2-1}
 \end{figure}
 
 \subsubsection{Ground}
 \paragraph{Docstring}
 This is the top-level function for the variational ground state search. Dependencies are \lstinline$Can$, \lstinline$ConvTest$, \lstinline$EffH$, \lstinline$Expect$, and \lstinline$SiteUpdate$. It is in principle the only function an end-user should have to deal with. Although mentioned here, explanation of the `standard format' of matrix product state used in this implementation is deferred until the main constructor function, \lstinline$CompMPS$, is described. Much of this function is devoted to book-keeping and reporting, though it does not produce a save file by itself except one after passing convergence threshold of \(1 \times 10^{-5}\). It does however, regularly output (hopefully) useful information to stdout. The rest of the function deals with the high level implementation of the variational search keeping track of what site is being updated, which direction the current sweep through the system is going, what the current energy of the system is, and how many sweeps have been completed. 
 \begin{lstlisting} 
 function [groundMPS, energyTracker] = Ground(init_mps, mpo, THRESHOLD, RUNMAX) \end{lstlisting}
 \begin{longtabu}{X[1,l]X[4,l]}
 \hline
 Return & \\ \hline
 \lstinline$groundMPS$ & \emph{\(N \times 1\) cell array}. Contains a matrix product state representing the approximate ground state of the system -- the solution state. Matrix product state is in the standard format used in this implementation. \\
 \lstinline$energyTracker$ & \emph{Dynamic complex double array}. Begins with a single element, the energy of the initial state evaluated as \(\langle \Psi_{\mathrm{init}}^{\mathrm{MPS}} | \hat{H} | \Psi_{\mathrm{init}}^{\mathrm{MPS}} \rangle \). The energy of the system is evaluated after each full sweep, and the result is stored as \lstinline$energyTracker(end + 1)$ enlarging the array up to a maximum size of \lstinline$RUNMAX$\(+ 1\), in the case where the energy does not converge. This array is used to monitor for convergence.\\ \hline
 Input & \\ \hline
 \lstinline$init_mps$ & \emph{\(N \times 1\) cell array}. Contains some initial state in the standard matrix product format used in this implementation. \\
 \lstinline$mpo$ & \emph{\(3 \times 1\) cell array}. Contains a matrix product operator representation of a Hamiltonian. \lstinline$mpo\{1\}$ contains the first site matrix product operator, \lstinline$mpo\{3\}$ contains the last site matrix product operator, and \lstinline$mpo\{2\}$ contains the bulk site matrix product operator. Each is a two dimensional complex double array, formed from \(d \times d\) blocks, as advised in section 6.1 of Ref~\cite{Schollwoeck11}. The specific size will be dependent on the Hamiltonian being represented, but if \(P\) is the number of blocks required then \lstinline$mpo\{1\}$ will be \(d \times Pd\), \lstinline$mpo\{2\}$ will be \(Pd \times Pd\), and \lstinline$mpo\{3\}$ will be \(Pd \times d\). \\
 \lstinline$THRESHOLD$ & \emph{Double}. The threshold below which the energy of the system will be considered to have converged. Should be real, greater than numerical error (whatever that may be on the relevant hardware), and less than \(1 \times 10^{-7}\) since the code won't check against this value until the energy has converged to at least that level.\\
 \lstinline$RUNMAX$ & \emph{Double}. The maximum number of site updates that will be performed before the code finishes regardless of convergence. Take care that there are \(N\) site updates per \emph{sweep}. Should be an integer value (much) greater than zero. Essentially this value is used to stop the code iterating forever when a calculation does not converge to the desired level.\\
 \hline
 \end{longtabu}
 \paragraph{Testing}
 \lstinline$GroundTest$ \& \lstinline$MPO_on-site+hop.mat$. Tests function against different local state space sizes, system sizes, and amounts of compression. Standard size, shape, type testing of return values. Checks that \lstinline$groundMPS$ is normalised. Loads simple Hamiltonian matrix product operator from \lstinline$MPO_on-site+hop.mat$ and checks that function correctly finds the \(|0_{1}0_{2}\ldots 0_{N} \rangle\) ground state.
 \paragraph{Improvements}
 \begin{itemize}
 \item MPO format could do with overhauling. In most cases the first site MPO tensor is the bottom row of the bulk MPO tensor, and the last site MPO tensor is the first column of the bulk MPO tensor, so cell array is unnecessary. Additionally, forcing the two-dimensional shape by joining virtual and physical indices is unnecessary and wastes time -- would be better to simply create a four-dimensional array. Modifying this would however require a modification to many of the lower level functions.
 \item Function only creates a save file at the convergence to \(1 \times 10^{-5}\) checkpoint. It should probably do so at the \(1 \times 10^{-7}\) checkpoint, and full convergence as well. The checkpoint thresholds should be inputs, and ideally optional.
 \end{itemize}
 
 \subsubsection{CompMPS} 
 \paragraph{Docstring}
 This is a constructor function for a random matrix product state. Its only dependency is \lstinline$MPSNorm$. It accepts some basic parameters of the physical system, and returns a normalised matrix product state of the correct size and shape in the standard format. The standard format is an \(N \times 1\) cell array, where each cell contains a three-dimensional complex double array. The third index is the physical index for that particular site, and so should run from \(1\) to \(d\). The first two indices are the first and second virtual indices for a particular site. The virtual indices should grow as we move through the system, so that the first site should be a \(1 \times d \times d\) array, and the second a \(d \times d^{2} \times d\) array, and so on until the limit set by \lstinline$COMPRESS$ is reached. At a certain point the virtual indices should begin shrinking again, such that the last site is a \(d \times 1 \times d\) array. This function and the other constructor, \lstinline$ProdMPS$ ensure this is the case.  
 \begin{lstlisting}
 function [complexMPS] = CompMPS(HILBY, L, COMPRESS) \end{lstlisting}
 \begin{longtabu}{X[1,l]X[4,l]}
 \hline
 Return & \\ \hline
 complexMPS & \emph{\lstinline$L$ \(\times 1\) cell array}. Contains a normalised matrix product state in the standard format with arrays that are at most \lstinline$COMPRESS$ \(\times\) \lstinline$COMPRESS$ \(\times\) \lstinline$HILBY$ in size. Each array is filled with random complex numbers generated using the MATLAB function \lstinline$rand$ in the form \lstinline$complexMPS\{site\} = rand(rowSize, colSize, HILBY) + 1i * rand(rowSize, colSize, HILBY)$. \\ \hline
 Input & \\ \hline
 \lstinline$HILBY$ & \emph{Double}. The size of the local state space, which we have previously referred to as \(d\). Should be a positive integer greater than 1.  \\
 \lstinline$L$ & \emph{Double}. The number of sites in the system, which we have previously referred to as \(N\). Should be a positive integer greater than 1. \\
 \lstinline$COMPRESS$ & \emph{Double}. The maximum allowed virtual dimension of the matrix prouct state tensors, which we have previously referred to as \(\chi_{\mathrm{max}}\). Note that if \lstinline$COMPRESS = 0$, it will be changed to the MATLAB value \lstinline$Inf$, meaning the matrix product state tensors will \emph{not} be compressed, and will reach a maximum size of \(d^{N/2}\).  \\
 \hline
 \end{longtabu}
 \paragraph{Testing}
 \lstinline$CompMPSTest$. Tests class, size, and shape of returned matrix product state. Uses a range of input values.
 \paragraph{Improvements}
 \begin{itemize}
 \item Testing should include checking that the returned matrix product state is properly normalised.
 \end{itemize}
 
 \subsubsection{ProdMPS}
 \paragraph{Docstring}
 This is a constructor function for a matrix product representation in the standard format of a simple product state. Its only dependence is \lstinline$MPSNorm$.
 \begin{lstlisting}
 function [matrices] = ProdMPS(stateArray, HILBY, COMPRESS) \end{lstlisting}
 \begin{longtabu}{X[1,l]X[4,l]}
 \hline
 Return & \\ \hline
 \lstinline$matrices$ & \emph{\(N \times 1\) cell array}. Contains a normalised matrix product state in the standard format, which represents the supplied product state. Formed by creating zero arrays of the correct size, and then replacing the appropriate matrices with the identity. \\ \hline
 Input & \\ \hline
 \lstinline$stateArray$ & \emph{\(N \times 1\) double array}. A column vector where each element refers to a particular site in the system, and contains the desired number state for that site. For example the three site state \(|1 1 0 \rangle\) would be supplied as \lstinline$[1; 1; 0]$. \\
 \lstinline$HILBY$ & \emph{Double}. The size of the local state space, which we have previously referred to as \(d\). Should be a positive integer greater than 1.  \\
 \lstinline$COMPRESS$ & \emph{Double}. The maximum allowed virtual dimension of the matrix prouct state tensors, which we have previously referred to as \(\chi_{\mathrm{max}}\). Note that if \lstinline$COMPRESS = 0$, it will be changed to the MATLAB value \lstinline$Inf$, meaning the matrix product state tensors will \emph{not} be compressed, and will reach a maximum size of \(d^{N/2}\).  \\
 \hline
 \end{longtabu}
 \paragraph{Improvements}
 \begin{itemize}
 \item Needs unit tests.
 \item Currently the code enforces that \lstinline$stateArray$ is a column vector, but there's no good reason for this.
 \end{itemize}
 
 \subsubsection{MPSNorm}
 \paragraph{Docstring}
 This function normalises a provided matrix product state. This is achieved by performing a QR decomposition on every site from the first, exactly as if one were making the state left-canonical. The major difference is the the final site in the system is also decomposed, and the final \(R\) matrix is simply discarded rather than being multiplied into the next site (as there is none). This ensures that \(\langle \Psi | \Psi \rangle = 1\).
 \begin{lstlisting}
 function [normalMPS] = MPSNorm(mps) \end{lstlisting}
 \begin{longtabu}{X[1,l]X[4,l]}
 \hline
 Return & \\ \hline
 \lstinline$normalMPS$ & \emph{\(N \times 1\) cell array}. Contains the normalised form of the input matrix product state, \lstinline$mps$.  \\ \hline
 Input & \\ \hline
 \lstinline$mps$ & \emph{\(N \times 1\) cell array}. Some presumably un-normalised matrix product state, in the standard format. \\
 \hline
 \end{longtabu}
 \paragraph{Testing}
 \lstinline$MPSNormTest$. Builds matrix product states using a range of system sizes and levels of compression, and then normalises them. Rebuilds the state vector and and calculates \(\langle \Psi | \Psi \rangle\). 
 \paragraph{Improvements}
 \begin{itemize}
 \item \lstinline$MPSNormTest$ calls \lstinline$CompMPS$, a function which itself depends on \lstinline$MPSNorm$. This should be avoided if possible. 
 \item \lstinline$MPSNormTest$ rebuilds the entire state vector which is never efficient, and may be avoidable. Worse, it depends on the debug function \lstinline$Rebuild$ to do so. This should be changed.
 \end{itemize}
 
 \subsubsection{Can}
 \paragraph{Docstring}
 
 \subsubsection{GrowBlock}
 
 \subsubsection{ConvTest}
 
 \subsubsection{EffH}
 
 \subsubsection{Expect}
 
 \subsubsection{GrowLeft}
 
 \subsubsection{GrowRight}
 
 \subsubsection{LCan}

 \subsubsection{RCan}
 
 \subsubsection{SiteUpdate}
 
\FloatBarrier 
 
 \subsection{Stationary State Implementation}