In this section we consider variational searches using Matrix Product States. We will first discuss the theoretical underpinnings of such a technique -- what we vary, and what we search for. I will then describe details pertaining to my own implementations of these techniques. There is no suggestion that these implementation details represent best practise, or are in any sense \emph{the right way} to perform these calculations. They are merely how \emph{I} approached the problem. All these implementations were written in and for MATLAB, and can at time of writing be found in repositories hosted at Ref~\cite{otb:githome}.
 
 \subsection{Theory}
It is well known that one can use the Rayleigh-Ritz Variational Technique to find an approximation to the lowest eigenvalue and corresponding eigenfunction of a Hermitian operator. Given a set of variational parameters upon which the eigenfunctions depend, one can move always to a lower eigenvalue, by minimising over one parameter at a time \cite{ArfWeb_RRVT, Gasiorowicz_RVT}. Consequently, we can find an approximation to the ground state of a system by minimising the expression,
\begin{equation}
E = \frac{\langle \psi (\bar{x}^{*}) | \hat{H} | \psi (\bar{x}) \rangle}{\langle \psi (\bar{x}^{*}) | \psi (\bar{x}) \rangle},
\label{eq:vs1-1}
\end{equation}
where E is the energy of the system, \(\hat{H}\) is a Hamiltonian, \(\psi\) is an approximation to the ground state, and \( \bar{x} \) is some set of \emph{variational parameters}. Equally, we can find an approximation to the stationary state of an open quantum system by minimising the expression,
\begin{equation}
\frac{\mathrm{d}\rho}{\mathrm{d}t} = \langle \rho(\bar{x}^{*}) | \hat{\mathcal{L}} | \rho(\bar{x}) \rangle,
\label{eq:vs1-2}
\end{equation}
where \(\hat{\mathcal{L}}\) is a Liouvillian, \(\rho\) is an approximation to the stationary state, and \(\bar{x}\) is again some set of variational parameters. For the purposes of this theoretical discussion we will focus on the ground state case as the two cases are very similar, but there are additional complexities in the stationary state search. 

When using Matrix Product States the set of variational parameters we employ are the individual site tensors. We shall discuss the search procedure as prescribed by Ulrich Schollw\"{o}ck's excellent review article \cite{Schollwoeck11}.

\begin{figure}
\centering
\includegraphics[width=0.8\linewidth]{\figpath/var_space}
\caption{A visual representation of a variational search using matrix product states. The purple background represents the total state space of the system, and the green oval is the part of that state space that can be represented by a matrix product state of some finite dimension. The orange star represents our desired solution state, and it is inaccessible to the matrix product state space. The black circle is the initial matrix product state, the black star is the nearest matrix product state approximation to the solution state, and the black squares are states through which the matrix product state transitions on its way to the solution state. The black dashed line represents a variational step -- optimisation over one or more of the variational parameters. The transitional states may or may not have some physical meaning in the context of the variational search depending on the specifics of the system being investigated. In general, however, if one wishes to know \emph{how} a system reaches the solution state a time evolution method should be used, not a variational search.}
\label{fig:vs1-1}
\end{figure}
 
 \subsection{Ground State Implementation}
 
 \subsection{Stationary State Implementation}