This thesis is structured in the following way. We will begin in \cref{chp:MBQOQS} by discussing many-body quantum physics, in particular the scaling problem, which makes it such a challenging field. We will then introduce the driven-dissipative Bose-Hubbard model, which we are particularly interested in investigating. We will give two examples of physical systems in which such a model can be implemented, paying special attention to the introduction of nonlinearity and dissipation to the system. We will then introduce the basics of open quantum systems, remarking on the Lindblad-form master equation. We limit ourselves to solving for the steady state of a system under the action of Lindblad dynamics, though the relaxation of the approximations inherent to the Lindblad master equation is itself an open and very interesting field of research. Finally in this chapter, we will briefly consider the practicalities of numerically solving for the steady state. We do so as a sort of primer for the following chapter.

In \cref{chp:NM} we will begin by introducing matrix product states -- one answer to the scaling problem. MPS are a way of representing quantum states which allow for the compression of states with low levels of entanglement. They were first introduced in the form in which we use them by Guifr\'{e} Vidal \cite{Vidal2003,Vidal2004}, though they are essentially an evolution of the DMRG method due to Steven White \cite{White1992,White1993}, and the analytical matrix product states previously used to study finitely correlated states \cite{Affleck1987}. We then introduce the matrix product operator, a way of representing a Hamiltonian or, more importantly as far as we are concerned, a Liouvillian, in a similar format to the matrix product state. The MPO is critical to the variational search technique, and our work owes most to the recent efforts of Cui \cite{Cui2015} and Mascarenhas \cite{Mascarenhas2015}, who extended the variational search approach from finding the ground state of a closed system, to finding the stationary state of an open one. One of, if not \emph{the}, key outputs of my PhD is a variational stationary state search code \cite{otb:gitVSSS} which was written for Matlab \cite{MATLAB}. The documentation for that code is included in this thesis as \cref{chp:mpostat}. We also, more briefly, mention time evolution methods using matrix product states. Although powerful, the variational search technique does not work for every system. Time evolution therefore still plays an important role in research presented later in the thesis.

Next in \cref{chp:DIM} we consider the first published research of this thesis, ``Dissipation-induced mobility and coherence in frustrated lattices'' \cite{Owen2017}. Here we considered a model of a geometrically frustrated lattice system, which was coupled to a dissipative environment in such a way that dissipative transport was possible. The first half of the paper considers the non-interacting regime, which could be solved using the Ehrenfest equations, while the second half considers a strongly interacting regime. This was the first attempt we made to use the variational search code on a `real' problem, and it was successful. We determined that dissipation enabled transport through the lattice in both the non-interacting, and interacting case, although strong interaction suppressed mobility. Designing an MPO which includes so many non-local terms is a little tricky, so the full MPO is included in \cref{chp:dimmpo} along with some explanatory notes, in the hope it may prove useful to others pursuing similar investigations.

Then in \cref{chp:DNLCA} we expand on work which has been submitted though not yet published, ``Localization to delocalization crosssover in a driven nonlinear cavity array'' \cite{Brown2018}. This was the principle scientific investigation of my PhD, however, this system was not amenable to solution using the variational search technique as we had hoped it might be. Instead we made use of TEBD, a matrix product state time evolution method. This was effective, though computationally costly. We were able to determine that a parametrically driven nonlinear cavity array, with dissipation that was larger for higher numbers of excitations on a site, exhibits a localized steady state analogous to the Mott insulator when the hopping rate between sites is low. At high hopping rates, we did not observe a superfluid-like state as in the equilibrium case, as long range coherences did not build up. \Cref{chp:rotframe,chp:adelim} are technical appendices containing derivations related to this project, which were too lengthy for inclusion in the article.

In \cref{chp:FW} we look at a new project which I have done some preliminary work on, which again features non-local dissipation. It is a biased spin chain, where the splitting of the two levels decreases from one end to the other. It has already shown some interesting results regarding dissipative transport. Such a system has previously been studied as a model of a biological photocell \cite{Fruchtman2016}, however that investigation was limited to the subspace of only one excitation. We consider a similar system in the many-body context. 

Finally, in \cref{chp:Conclusion} we indulge in some conjecture as to the future direction of both that project, and of the variational search code, in order to conclude the thesis by casting our eyes forward. 