In this thesis we began by introducing the scaling problem, the fundamental difficulty of many-body quantum physics, and then went on to describe in some detail the driven Bose-Hubbard model which the research presented focuses on. In addition we provide some physical motivation for such a model. Next we introduced some standard tools of open quantum systems -- the Lindblad-form master equation, and the Liouvillian matrix. Ultimately the problem of finding the stationary state of a driven-dissipative system is the same as finding the ground state of a closed system, except that the matrix is much larger and not Hermitian! In the next chapter we considered one method for overcoming the challenge this presents computationally -- matrix product states (and operators). We go in to a lot of detail on one particular technique, the variational search, and for good reason. The \lstinline$mpostat$ code found in the online repository hosted at \cite{otb:gitVSSS}, and available under an open source license, implements this technique. It is one of the principle outputs of my PhD, and the documentation is included in this thesis in \cref{chp:mpostat}. Like any technique, the variational search has both advantages and disadvantages. It is generally quicker than time-evolution methods, and it is easy to encode long range interactions in the matrix product operator format used to specify the system to be solved. On the other hand, it is memory hungry, and the result is not guaranteed to be physical, so it must be used with care. Nevertheless, as is shown in reference~\cite{Owen2017}, in the work presented in this chapter, and in work done by others \cite{Cui2015,Mascarenhas2015}, the technique is powerful enough to push at and expand the boundaries of what is possible in this challenging field. 

We then considered the two research papers produced during my PhD, reference~\cite{Owen2017} (published) and reference~\cite{Brown2018} (submitted), and expanded on what is available in their original format. In the first, we considered a geometrically frustrated lattice system where each site in one of the two sublattices was dissipatively coupled to an independent bath. We found that doing so enabled transport through the lattice in both the non-interacting, and interacting regimes, although strong interaction did suppress mobility. In the second we considered an array of nonlinear cavities with a parametric coherent drive to the doubly excited state, and dissipation which increased with number of excitations in the cavity. We showed that in this system there is a crossover in the steady state from a localized state, analogous to the Mott insulator, to a delocalized state. We found that in contrast to the equilibrium case, first order coherence does not become long ranged in this system, and we do not see a superfluid state form. Finally, in the previous chapter we have discussed one of the new projects that we have begun working on, and which will make use of the variational search code. This has shown some promising initial results, with a possible transition from a phase in which incoherent processes dominate transport, to one in which coherent processes dominate. 

As for the code itself, future directions for improvement would include making it more flexible, while not compromising its focus. Other libraries such as the TNT Library \cite{TNTlib,Al-Assam2017} implement matrix product state methods more generally, and do it well, and it would be unwise to attempt to replicate that effort by allowing \lstinline$mpostat$ to bloat. One specific recommendation would be to simplify the data structures by reimplementing matrix product states and operators as classes rather than cell arrays, which could make it easier for users to write the MPO for any given system, and would make it easier to make the code more flexible. If I were to have another four years to work on the code, that is one of the first things I would do! Other than that, I would continue to make improvements to the top-level control logic and reporting, so as to make the code more user-friendly, and I would introduce some automatic testing of the results. 

Improvements to the code would in turn contribute to investigating the physics of more computationally challenging systems. We could, for example consider higher dimensional systems. This would be an interesting way to extend the investigation of the localization to delocalization transition in a driven nonlinear cavity array, as it would allow us to consider a more complex driving scheme. A four-level scheme for instance would make it easier to reach the Mott insulator state and we could investigate integer filling of greater than one particle per site. It would also allow us to investigate the same system at a higher hopping rate, where we might find the long range coherence we had originally hoped to find. We could also, in general, consider larger systems which would serve to ensure that we eliminate boundary effects, and could helps us uncover emergent many-body behaviour. Improving the code and extending the size of the lattice would mean that we could consider longer range dissipative interactions in the frustrated lattice system -- though that would certainly be a challenging set of calculations.

It is my hope that \lstinline$mpostat$ will continue to be used to investigate challenging systems in the field of driven dissipative many-body quantum systems, and will continue to be improved, and I am pleased that it has already proven useful. That is all, thank you for reading.