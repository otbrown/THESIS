In this chapter we will discuss the article `Localization to delocalization crossover in a driven nonlinear cavity array' \cite{Brown2018}. In this work we studied nonlinear cavity arrays where the dissipation rate in each cavity increased with the excitation number. It was shown that coherent parametric driving such arrays into states with commensurate filling -- a non-equilibrium analogue of the Mott insulating state. We explore the boundaries of this Mott insulating phase and the crossover to a delocalized phase with spontaneous first order coherence. This crossover is similar to the equilibrium Mott insulator to superfluid phase transition, but we also find marked differences in the phase-diagrams. In particular, in this system the off diagonal order does not become long range.  

As ever we will begin by discussing the motivation and theoretical background to the work, followed by a thorough description of the model. We will then consider an approach which did \emph{not} work, using a mean-field approximation. Finally we will go through the results, mostly from TEBD calculations \cite{Vidal2003}, which are presented in the article. Finally we include a derivation of a two-level approximation to the master equation, and make our conclusions.

This was the primary research project of my PhD, and it was originally hoped that it could be used to test out the \lstinline$mpostat$ code. Unfortunately, this was a challenging investigation, and the master equation proved resistant to numerical solution. Ultimately, the matrix dimensions required for solution by variational search were simply too high. Nevertheless, we were able to determine the steady state using time-evolution methods. 

\section{Introduction}
Photons are not usually conserved in light-matter interactions. Consequently, there is no chemical potential for photons, and the rich vein of many-body quantum effects in equilibrium systems is seemingly lost to photonics. Some exceptions, where the concept of an effective chemical potential can be meaningfully applied to photons, include photon emission in semiconductors \cite{Wurfel1982}, photons in a cavity that couple to excitons and thermalize \cite{Keeling2007,Eastham2001,Carusotto2013}, and photons interacting with a nonlinear medium that form a Bose-Einstein condensate \cite{Kasprzak2006,Klaers2010}. Settings where light-matter interactions can mediate strong photon-photon have garnered significant interest recently, as these allow generation of matter-like phases including photonic fluids \cite{Carusotto2013,Vocke2015} and strongly correlated phases \cite{Hartmann2008,Hartmann2016,Noh2017}.

Since photons are bosons, a key question for many-body phenomena in strongly interacting photon or polariton systems is whether a phase transition can be observed from a Mott insulator to a superfluid \cite{Hartmann2007} as it is in Bose-Einstein condensates \cite{Fisher1989,Jaksch1998,Greiner2002}. Phase diagrams of equilibrium photonic or polaritonic systems have previously been studied by introducing a chemical potential. How such a thing could be physically realised remains an open question \cite{Greentree2006,Koch2009,DeLeeuw2015,Hartmann2016}. In any case, we consider the non-equilibrium setting a more natural one in which to study such systems, given the limited lifetime of photons trapped in a cavity. One approach to that is to use auxiliary systems and specific driving mechanisms to generate an \emph{effective} chemical potential for photons \cite{Hafezi2015,Ma2017,Lebreuilly2017}, allowing one to explore the phase diagram \cite{Biella2017}.

In this work we show that a Mott insulator phase can be generated in a dissipative nonlinear cavity array using only a coherent parametric drive directly applied to the cavities. We thus explore the crossover from this Mott insulating state to a delocalized phase with incommensurate filling \cite{Hartmann2010,LeBoite2013,Jin2013,Abbarchi2013,Raftery2014,Altman2015,Dagvadorj2015}, which exhibits first order coherence between lattice sites. A key feature of the Mott insulator phase is that there is an integer number of excitations on every site, and that number fluctuations are strongly suppressed. However, this cannot be achieved in a nonlinear resonator array which is coherently driven at the frequency of a single excitation. The Mott insulator phase is expected in the limit of very strong nonlinearity, and very weak coupling between sites. In this regime each lattice site may be approximated as a two level system where filling cannot exceed half -- a consequence of the depopulation of the upper state under a coherent drive. Additionally, the phase relation between the coherent drive on different lattice sites is fixed. As a result, any phase-coherence between excitations on distant sites could be said to be inherited from the drive \cite{Ruiz-Rivas2014}, and it is unclear whether such coherence forms spontaneously at equilibrium \cite{Fisher1989,Jaksch1998,Greiner2002}.

It is for these reasons that we consider instead a parametric coherent drive, which resonantly drives each cavity from empty to doubly-excited \cite{Ma2017,Savona2017}, but is detuned from all other transitions. We also include a cascade of decay processes, where the decay from the doubly-excited state to a single excitation state, \(\gamma_{1}\), is much faster than the decay of a single excitation state to the empty state, \(\gamma_{0}\). This arrangement results in a very high probability for the single excitation state on each lattice site to be stationary -- the probability approaches unity as \(\gamma_{0}/\gamma_{1} \rightarrow 1\). The drive and decay process combine to form an effective incoherent drive from the empty state to the single excitation state. As any excitations in level \(|1\rangle\) are generated via the fast decay rate \(\gamma_{1}\), they are insensitive to the coherent nature of the drive. Thus allowing us to attribute any first-order correlations we find to the formation of a superfluid component. \Cref{fig:dnlca1-1} shows a sketch of a two site model.

\begin{figure}[ht!]
\centering 
\includegraphics[width=0.8\linewidth]{\figpath/DNLCA_model}
\caption{\label{fig:dnlca1-1}A diagram of the two site model, showing states with zero (\(|0\rangle\)), one (\(|1\rangle\)), and two (\(|2\rangle\)) excitations in each cavity, as well as the key parameters. The two sites are coupled by a hopping rate, \(J\), there is a coherent parametric drive on each site with amplitude, \(\Omega\), and there are two dissipative transition rates, \(\gamma_{1} \gg \gamma_{0}\).}
\end{figure}

\section{The model}

\section{Mean-field approach}

\section{Small anharmonic system}

\section{Large anharmonic system}

\section{Large harmonic system}

\section{Adiabatic elimination}

\section{Conclusions}
