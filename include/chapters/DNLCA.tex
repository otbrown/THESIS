In this chapter we will discuss the article `Localization to delocalization crossover in a driven nonlinear cavity array' \cite{Brown2018}. In this work we studied nonlinear cavity arrays where the dissipation rate in each cavity increased with the excitation number. It was shown that with a coherent parametric drive such arrays can be driven in to states with commensurate filling -- a non-equilibrium analogue of the Mott insulating state. We explore the boundaries of this Mott insulating phase and the crossover to a delocalized phase with spontaneous first order coherence. This crossover is similar to the equilibrium Mott insulator to superfluid phase transition, but we also find marked differences in the phase-diagrams. In particular, in this system, the off-diagonal order does not become long range.  

As ever we will begin by discussing the motivation and theoretical background to the work, followed by a thorough description of the model. We will then go through the results, mostly from TEBD calculations \cite{Vidal2003}, which are presented in the article. Finally we include a derivation of a two-level approximation to the master equation, and make our conclusions.

This was the primary research project of my PhD, and it was originally hoped that it could be used to test out the \lstinline$mpostat$ code. Unfortunately, this was a challenging investigation, and the master equation proved resistant to numerical solution. Ultimately, the matrix dimensions required for solution by variational search were simply too high. Nevertheless, we were able to determine the steady state using time-evolution methods. 

\section{Introduction}

Photons are not usually conserved in light-matter interactions. Consequently, there is no chemical potential for photons, and the rich vein of many-body quantum effects in equilibrium systems is seemingly lost to photonics. Some exceptions, where the concept of an effective chemical potential can be meaningfully applied to photons, include photon emission in semiconductors \cite{Wurfel1982}, photons in a cavity that couple to excitons and thermalize \cite{Keeling2007,Eastham2001,Carusotto2013}, and photons interacting with a nonlinear medium that form a Bose-Einstein condensate \cite{Kasprzak2006,Klaers2010}. Settings where light-matter interactions can mediate strong photon-photon have garnered significant interest recently, as these allow generation of matter-like phases including photonic fluids \cite{Carusotto2013,Vocke2015} and strongly correlated phases \cite{Hartmann2008,Hartmann2016,Noh2017}.

Since photons are bosons, a key question for many-body phenomena in strongly interacting photon or polariton systems is whether a phase transition can be observed from a Mott insulator to a superfluid \cite{Hartmann2007} as it is in Bose-Einstein condensates \cite{Fisher1989,Jaksch1998,Greiner2002}. Phase diagrams of equilibrium photonic or polaritonic systems have previously been studied by introducing a chemical potential. How such a thing could be physically realised remains an open question \cite{Greentree2006,Koch2009,DeLeeuw2015,Hartmann2016}. In any case, we consider the non-equilibrium setting a more natural one in which to study such systems, given the limited lifetime of photons trapped in a cavity. One approach to that is to use auxiliary systems and specific driving mechanisms to generate an \emph{effective} chemical potential for photons \cite{Hafezi2015,Ma2017,Lebreuilly2017}, allowing one to explore the phase diagram \cite{Biella2017}.

In this work we show that a Mott insulator phase can be generated in a dissipative nonlinear cavity array using only a coherent parametric drive directly applied to the cavities. We thus explore the crossover from this Mott insulating state to a delocalized phase with incommensurate filling \cite{Hartmann2010,LeBoite2013,Jin2013,Abbarchi2013,Raftery2014,Altman2015,Dagvadorj2015}, which exhibits first order coherence between lattice sites. A key feature of the Mott insulator phase is that there is an integer number of excitations on every site, and that number fluctuations are strongly suppressed. However, this cannot be achieved in a nonlinear resonator array which is coherently driven at the frequency of a single excitation. The Mott insulator phase is expected in the limit of very strong nonlinearity, and very weak coupling between sites. In this regime each lattice site may be approximated as a two level system where filling cannot exceed half -- a consequence of the depopulation of the upper state under a coherent drive. Additionally, the phase relation between the coherent drive on different lattice sites is fixed. As a result, any phase-coherence between excitations on distant sites could be said to be inherited from the drive \cite{Ruiz-Rivas2014}, and it is unclear whether such coherence forms spontaneously at equilibrium \cite{Fisher1989,Jaksch1998,Greiner2002}.

It is for these reasons that we consider instead a parametric coherent drive, which resonantly drives each cavity from empty to doubly-excited \cite{Ma2017,Savona2017}, but is detuned from all other transitions. We also include a cascade of decay processes, where the decay from the doubly-excited state to a single excitation state, \(\gamma_{1}\), is much faster than the decay of a single excitation state to the empty state, \(\gamma_{0}\). This arrangement results in a very high probability for the single excitation state on each lattice site to be stationary -- the probability approaches unity as \(\gamma_{0}/\gamma_{1} \rightarrow 1\). The drive and decay process combine to form an effective incoherent drive from the empty state to the single excitation state. As any excitations in level \(|1\rangle\) are generated via the fast decay rate \(\gamma_{1}\), they are insensitive to the coherent nature of the drive. Thus allowing us to attribute any first-order correlations we find to the formation of a superfluid component. This could be experimentally achieved through Purcell enhancement \cite{Purcell1946,Fox_Purcell} of the relaxation on a specific transition, through coupling to a lossy resonator which is resonant with this particular transition \cite{Bienfait2016}. \Cref{fig:dnlca1-1} shows a sketch of a two site model.

\begin{figure}[ht!]
\centering 
\includegraphics[width=0.8\linewidth]{\figpath/DNLCA_model}
\caption{\label{fig:dnlca1-1}A diagram of the two site model, showing states with zero (\(|0\rangle\)), one (\(|1\rangle\)), and two (\(|2\rangle\)) excitations in each cavity, as well as the key parameters. The two sites are coupled by a hopping rate \(J\), there is a coherent parametric drive on each site with amplitude \(\Omega\), and there are two dissipative transition rates, \(\gamma_{1} \gg \gamma_{0}\).}
\end{figure}

\section{The model}

We consider a system of \(N\) coupled nonlinear cavities in a one-dimensional array, governed by a Bose-Hubbard Hamiltonian, with an additional term for the parametric coherent driving scheme described above. We move into a rotating frame and apply the rotating wave approximation, which is shown in detail in \cref{chp:rotframe}, and set \(\hbar = 1\) to work in units of frequency. Our Hamiltonian then is,
\begin{equation}
	\mathcal{H} = \mathcal{H}_{0} + \mathcal{H}_{J} + \mathcal{H}_{\Omega},
	\label{eq:dnlca2-1}
\end{equation}
where,
\begin{align}
	\mathcal{H}_{0} &= \sum_{j} \left[ \Delta \hat{a}_{j}^{\dagger}\hat{a}_{j} + \frac{U}{2}\hat{a}_{j}^{\dagger}\hat{a}_{j}^{\dagger}\hat{a}_{j}\hat{a}_{j}\right], \label{eq:dnlca2-2} \\
	\mathcal{H}_{J} &= -J \sum_{j}\left[ \hat{a}_{j}\hat{a}_{j+1}^{\dagger} + \hat{a}_{j}^{\dagger}\hat{a}_{j+1} \right], \label{eq:dnlca2-3} \\
	\mathcal{H}_{\Omega} &= \sum_{j} \left[\frac{\Omega}{\sqrt{2}}\hat{a}_{j}^{\dagger}\hat{a}_{j}^{\dagger} + \frac{\Omega^{*}}{\sqrt{2}}\hat{a}_{j}\hat{a}_{j} \right], \label{eq:dnlca2-4}
\end{align}
where \(\Delta = \omega - \omega_{L}/2\) is the detuning between the drive laser frequency \(\omega_{L}\), and the cavity frequency \(\omega\), \(U\) is the interaction strength, \(J\) is the hopping rate between sites, and \(\Omega\) is the amplitude of the drive laser. The drive laser is tuned to the two excitation frequency \(\omega_{L} = 2\omega + U\), so the detuning \(\Delta = -U/2\). 

The dissipative environment we consider is characterized by a cascade of dissipation rates, so the dissipation rate \(\gamma_{m}\) from \(|m+1\rangle \rightarrow |m\rangle\) is greater than the dissipation rate \(\gamma_{n}\) from \(|n+1\rangle \rightarrow |n\rangle\) where \(m > n\). We describe this dissipation with a Lindblad-form master equation,
\begin{equation}
	\dot{\rho} = -i\left[\mathcal{H}, \rho\right] + \sum_{m \geq 0} \mathcal{D}_{m}[\rho],
	\label{eq:dnlca2-5}
\end{equation}
where,
\begin{equation}
	\mathcal{D}_{m} [\rho] = \frac{\gamma_{m}}{2} \sum_{j}^{N} \left[ 2\kappa_{m,j}\rho\kappa_{m,j}^{\dagger} - \left\{\kappa_{m,j}^{\dagger}\kappa_{m,j}, \rho \right\}\right],
	\label{eq:dnlca2-6}
\end{equation}
where the jump operators \(\kappa_{m,j} = |m_{j} \rangle \langle m+1_{j}|\). Note that this model assumes that the dissipation is dominated by single particle losses and would reduce to the standard dissipator \(\frac{\gamma}{2} \sum_{j}\left[2\hat{a}_{j}\rho\hat{a}_{j}^{\dagger} - \{\hat{a}_{j}^{\dagger}\hat{a}_{j}, \rho\}\right]\) in the limit where all relaxation rates become equal, \(\gamma_{m} = \gamma\). Furthermore, we assume that \(U \gg \Omega\) so that the occupation of levels \(|m\rangle\) with \(m > 2\) is negligible, allowing us to truncate our description to the subspace of at most two excitations on each site. 

\section{Small anharmonic system}

We first considered a small exactly solvable model with just three sites and periodic boundary conditions, so that there is a hopping term of the form \(-J(\hat{a}_{3}\hat{a}_{1}^{\dagger} + \hat{a}_{3}^{\dagger}\hat{a}_{1})\) included in the model. This model should tell us something about the behaviour of larger systems in the limit of a small hopping rate between sites \(J / \gamma_{1} \ll 1\), where we expect long range correlations to be absent. Additionally, we extended the local state space to include up to three excitations per site, in order to test the validity of our truncation to three levels. The stationary state was determined by forming the Liouvillian matrix,
\begin{align}
	\hat{L} = &\mathbb{I} \otimes -i\mathcal{H} + i\mathcal{H}^{T} \otimes \mathbb{I} \notag \\ 
	&+ \sum_{m=0,j=1}^{m=2,j=3}\frac{\gamma_{m}}{2}\left[ \kappa_{m,j}^{*} \otimes 2\kappa_{m,j} - \mathbb{I} \otimes \kappa_{m,j}^{\dagger}\kappa_{m,j} - \kappa_{m,j}^{\dagger}\kappa_{m,j} \otimes \mathbb{I}\right],
	\label{eq:dnlca3-1}
\end{align}
and then replacing the bottom row of the Liouvillian with one which enforces the trace norm condition, \(\sum_{j} \rho_{j,j} = 1\). We can then find the stationary state numerically by solving the system of coupled linear equations,
\begin{align}
	 \hat{L}|\rho \rangle \rangle &= \bar{S}, \notag \\
	 \implies |\rho \rangle \rangle &= \hat{L}^{-1}\bar{S},
	 \label{eq:dnlca3-2}
\end{align}
where \(|\rho\rangle\rangle\) is the vectorised density matrix, and \(\bar{S}\) is a solution vector which is all-zero except for the last element which corresponds to the trace norm condition.

The equilibrium phase diagram for the Bose-Hubbard model is typically parameterized by the chemical potential and the hopping rate between sites. In our non-equilibrium case, the drive strength and dissipation rates balance out to create an effective chemical potential, so we explore a phase diagram parameterized by the drive strength and hopping rate. 

\Cref{fig:dnlca3-1} shows the density \(\langle \hat{n}_{2} \rangle\), and its variance \(\langle \hat{n}_{2}^{2} \rangle - \langle \hat{n}_{2} \rangle^{2}\) for the second site in this translation invariant system. Both are plotted against the drive strength \(\Omega\), and the hopping rate \(J\), for the reasons given above. In \cref{fig:dnlca3-1a} there is a region bounded by a black line, in which the density is unity, and the density variance is much less than one. This means there is a stationary phase of our model with very similar properties to the Mott insulator phase, though the shape differs somewhat from that found in equilibrium systems \cite{Rossini2007}. We note that even at the strongest driving we consider here, the density barely exceeds unity, validating our decision to truncate to the subspace of at most two excitations per site in calculations on larger systems.

\begin{figure}[ht]
	\subfloat[\label{fig:dnlca3-1a}]{\includegraphics[width=0.49\linewidth]{\figpath/threeSiteDensity_int100}} \hfill
	\subfloat[\label{fig:dnlca3-1b}]{\includegraphics[width=0.49\linewidth]{\figpath/threeSiteDensityVar_int100}}
	\caption{\label{fig:dnlca3-1} (a) The density of one site in a translation invariant three site system, plotted against drive strength \(\Omega\), and coupling strength \(J\). The area above the red line has a density of \(1 \pm 0.1\). (b) The variance, \(\langle \hat{n}_{2}^{2} \rangle - \langle \hat{n}_{2} \rangle^{2}\) over the same parameter range. For this calculation, \(\gamma_{2} = 10\), \(\gamma_{1} = 1\),\(\gamma_{0} = 0.1\), \(U=100\), and so \(\Delta = -50\). The area bounded by the red line has a variance of \(\leq 0.2\). The area bounded by the black line in panel (a) has both unit density and a density variance \(\ll 1\), which we identify with the Mott Insulator phase.}
\end{figure}

To further explore the boundaries of this phase and its predicted crossover to a highly-correlated delocalized phase, we next considered a much larger lattice.

\section{Large anharmonic system}

Here we consider a fifteen site lattice with open boundaries, and up to two excitations per site. We find the stationary state by representing the system with a density matrix product operator, then time-evolving it using the TEBD method \cite{Vidal2004} until convergence. \Cref{fig:dnlca4-1} shows the density of the middle site and its variance plotted against drive strength and hopping rate. It can be seen that for strong enough drive density is near unity, and as with the small system, the variance is much less than one at low hopping rates. The non-equilibrium Mott insulator-like phase survives the increase in system size.

\begin{figure}[ht]
	\subfloat[\label{fig:dnlca4-1a}]{\includegraphics[width=0.49\linewidth]{\figpath/anhDensity}} \hfill
	\subfloat[\label{fig:dnlca4-1b}]{\includegraphics[width=0.49\linewidth]{\figpath/anhDensityVar}}
	\caption{\label{fig:dnlca4-1} (a) The density of the middle site in a fifteen site system, plotted against drive strength \(\Omega\), and coupling strength \(J\). (b) The variance in the density over the same parameter range. For this calculation \(\gamma_{1} = 1\), \(\gamma_{0} = 0.1\), \(U = 20\), and so \(\Delta = -10\).}
\end{figure}

In the region where the local density fluctuations are larger we are interested in determining if a transition to a superfluid state occurs, as is the case in equilibrium systems. Such a transition would be signified by an increase in long range first order coherence. Since the single excitation state in our system is populated by incoherent decay from the upper state, this state is insensitive to the coherence of the drive. The combination of the coherent drive to the upper state and incoherent decay creates an effective incoherent pump process to the middle state. This interpretation is borne out by an adiabatic elimination of the upper state, which is shown in detail in \cref{chp:adelim}. We can therefore attribute any first-order correlations we observe to the formation of a superfluid component. Such correlations are quantified by the normalised \(g^{(1)}\)-function,
\begin{equation}
	g^{(1)}(i,j) = \frac{\langle \hat{a}_{i}^{\dagger} \hat{a}_{j} \rangle}{\sqrt{\langle \hat{n}_{i} \rangle \langle \hat{n}_{j} \rangle}}.
	\label{eq:dnlca4-1}
\end{equation}

\begin{figure}[ht]
	\subfloat[\label{fig:dnlca4-2a}]{\includegraphics[width=0.49\linewidth]{\figpath/anhG1}} \hfill
	\subfloat[\label{fig:dnlca4-2b}]{\includegraphics[width=0.49\linewidth]{\figpath/anhCorrLength}}
	\caption{\label{fig:dnlca4-2} The first order correlation \(g^{(1)}(i,j)\) and the correlation length $\lambda$. For this calculation, \(\gamma_{1} = 1\), and \(\gamma_{0} = 0.1\). The interaction strength, \(U = 20\), and so \(\Delta = -10\). The first order correlation is plotted for a range of coupling strengths at a fixed drive strength, \(\Omega = 5\). The correlation length was determined by an \(\exp(- |j-j_{0}|/\lambda)\) fit to the \(g^{(1)}\) data.}
\end{figure}

It can be seen in \cref{fig:dnlca4-2a} that the build up of first order correlations occurs as the hopping rate increases. On the other hand, the range of these correlations does not increase monotonically with \(J\), but instead reaches a peak, and then decreases. We performed an exponential fit to the \(g^{(1)}\) data of the form,
\begin{equation}
	g^{(1)}(j) = \mathrm{e}^{-\frac{|j-j_{0}|}{\lambda}},
	\label{eq:dnlca4-2}
\end{equation}
where \(j\) labels each lattice site, \(j_{0}\) is the site on which correlations with other sites are measured (chosen to be the the middle site of the system), and \(\lambda\) is the correlation length. The result of this fit is shown in \cref{fig:dnlca4-2b}. We attribute the non-monotonic behaviour of the correlation length to competition between tunneling processes and dephasing processes. An increase in the hopping rate \(J\) enhances the long range coherence, but also enhances the local density fluctuations, resulting in more occupation of the doubly excited state. As double occupation increases, so does the contribution of the fast dissipation mechanism governed by the fast decay rate \(\gamma_{1}\), thus the ehanced dephasing.

Another characteristic of the equilibrium Mott insulator state is incompressibility -- the suppression of two-excitation coincidences. This effect can be quantified by the normalised \(g^{(2)}\)-function,
\begin{equation}
	g^{(2)}(i,j) = \frac{\langle \hat{a}_{i}^{\dagger}\hat{a}_{j}^{\dagger}\hat{a}_{j}\hat{a}_{i} \rangle}{\sqrt{\langle \hat{n}_{i} \rangle \langle \hat{n}_{j} \rangle}}.
	\label{eq:dnlca4-3}
\end{equation}
\Cref{fig:dnlca4-3} shows that the on-site density correlations are fully suppressed in the low hopping rate regime, and increase monotonically across the region.

\begin{figure}[ht]
	\centering
	\includegraphics[width=0.49\linewidth]{\figpath/anhG2}
	\caption{\label{fig:dnlca4-3} Second order correlation between sites, \(g^{(2)}(i,j)\) plotted against coupling strength. For this calculation, drive strength \(\Omega = 5\), \(\gamma_{1} = 1\), and \(\gamma_{0} = 0.1\). The interaction strength, \(U=20\), and so \(\Delta = -10\).}
\end{figure}

\section{Large harmonic system}

To further explore this class of systems we considered a lattice with \(\Delta = -U/2 = 0\), the harmonic regime. Though the case for truncation to the subspace of at most two excitations per site is weaker in this regime, we may rely on dissipation to ensure that no significant population will gather in energy levels above the doubly excited state in the steady state. As such, we again make this truncation, and use a TEBD code to find the steady state of an 11-site system with open boundary conditions.

In \cref{fig:dnlca5-1} we show the density and density variance of the middle site plotted, as with the anharmonic calculations, against drive strength \(\Omega\) and hopping rate \(J\). It can be seen that we again find a region where the excitation density is commensurate with the lattice size, and the variance is much less than one, indicative of the Mott insulator state. In contrast to the anharmonic results, this region is shifted in parameter space to a higher drive strength.

\begin{figure}[ht]
	\subfloat[\label{fig:dnlca5-1a}]{\includegraphics[width=0.49\linewidth]{\figpath/DensityScan}} \hfill
	\subfloat[\label{fig:dnlca5-1b}]{\includegraphics[width=0.49\linewidth]{\figpath/DensityVarScan}}
	\caption{\label{fig:dnlca5-1} Plots of the density on the middle site of an eleven site system, and its variance, plotted against drive strength \(\Omega\), and coupling strength \(J\). For this calculation, \(\gamma_{1} = 1\), and \(\gamma_{0} = 0.1\).}
\end{figure} 

In \cref{fig:dnlca5-2,fig:dnlca5-3} we show the correlation length, and first and second order correlations for a fixed drive strength. The principle difference between the correlations in the harmonic and anharmonic case is the presence of peaks and troughs on alternating sites in the first-order correlations in the harmonic case. This can be understood from the momentum basis representation of the master equation,
\begin{align}
	\mathcal{H}_{b} = &\sum_{k=0}^{N-1} \left[ \frac{\Omega}{\sqrt{2}}\hat{b}_{k}^{\dagger}\hat{b}_{N-k}^{\dagger} + \frac{\Omega^{*}}{\sqrt{2}}\hat{b}_{k}\hat{b}_{N-k} \right] + \sum_{k=0}^{N-1} \left[\Delta - 2J\cos\left(\frac{2\pi k}{N}\right)\right]\hat{b}_{k}^{\dagger}\hat{b}_{k} \notag \\ 
	&+ \sum_{(j,k,l,m)} \left[ \frac{U}{N} \hat{b}_{j}^{\dagger}\hat{b}_{k}^{\dagger}\hat{b}_{l}\hat{b}_{m} \right], 
	\label{eq:dnlca5-1}
\end{align}
where,
\begin{equation}
	\hat{a}_{n} = \frac{1}{\sqrt{N}} \sum_{k=0}^{N-1} \mathrm{e}^{i\frac{2\pi n}{N}k}\hat{b}_{k},
	\label{eq:dnlca5-2}
\end{equation}
and we have assumed periodic boundary conditions. The notation \((j,k,l,m)\) indicates that the indices range from \(0\) to \(N-1\) and follow the condition \(l+m-j-k = nN\), where \(n\) is some integer. When \(\Delta = 0\), as it does here, the detuning \(\Delta - 2J\cos (2 \pi k/N)\) is zero for modes with \(k = nN/4\). The drive is therefore resonant to these modes, and these momenta determine the correlation profile. In the anharmonic case the detuning \(\Delta\) is large compared to \(2J\) and the mode with \(k=0\) is closest to resonance with the drive.

\begin{figure}[ht]
	\subfloat[\label{fig:dnlca5-2a}]{\includegraphics[width=0.49\linewidth]{\figpath/g1}} \hfill
	\subfloat[\label{fig:dnlca5-2b}]{\includegraphics[width=0.49\linewidth]{\figpath/CorrLength}}
	\caption{\label{fig:dnlca5-2} The first order correlation \(g^{(1)}(i,j)\) and the correlation length $\lambda$. For this calculation, \(\gamma_{1} = 1\), and \(\gamma_{0} = 0.1\). The first order correlation is plotted for a range of coupling strengths at a fixed drive strength, \(\Omega = 5.5\). The correlation length was determined by an \(\exp(- |j-j_{0}|/\lambda)\) fit to the \(g^{(1)}\) data.}
\end{figure}
%
\begin{figure}[ht]
	\centering
	\includegraphics[width=0.49\linewidth]{\figpath/g2}
	\caption{\label{fig:dnlca5-3} Second order correlation between sites, \(g^{(2)}(i,j)\), plotted against coupling strength. For this calculation, drive strength \(\Omega = 5.5\), \(\gamma_{1} = 1\), and \(\gamma_{0} = 0.1\).}
\end{figure}

\section{Conclusions}

In conclusion, we have shown that in the model we describe, a nonlinear cavity array with a coherent parametric drive to the doubly excited state, and a dissipative regime in which higher excitation numbers decay more rapidly than lower excitation numbers, a stationary state exists with similar properties to the Mott insulator. Moreover we have shown that a crossover to a delocalized phase with spontaneous first order coherence can be found can be found by increasing the hopping rate between sites. Unlike the equilibrium case, where we would find a superfluid phase, the first order coherence does not become long range, and indeed decreases with increasing hopping rate after reaching a peak. We attribute this behaviour to dephasing processes becoming more prevalent as the number fluctuations increase as a result of the larger hopping rate. 