In this chapter we will discuss the article `Localization to delocalization crossover in a driven nonlinear cavity array' \cite{Brown2018}. As ever we will begin by discussing the motivation and theoretical background to the work, followed by a thorough description of the model. We will then consider an approach which did \emph{not} work, using a mean-field approximation. Finally we will go through the results, mostly from TEBD calculations \cite{Vidal2003}, which are presented in the article. Finally we include a derivation of a two-level approximation to the master equation, and make our conclusions.

This was the primary research project of my PhD, and it was originally hoped that it could be used to test out the \lstinline$mpostat$ code. Unfortunately, this was a challenging investigation, and the master equation proved resistant to numerical solution. Ultimately, the matrix dimensions required for solution by variational search were simply too high. Nevertheless, we were able to determine the steady state using time-evolution methods. 

\section{Introduction}

\section{The model}

\section{Mean-field approach}

\section{Small anharmonic system}

\section{Large anharmonic system}

\section{Large harmonic system}

\section{Adiabatic elimination}

\section{Conclusions}
