In this chapter we will discuss my contributions to the article `Dissipation-induced mobility and coherence in frustrated lattices' \cite{OBH17}. We will begin by discussing the motivation and theoretical background to the investigation, then we will discuss in some detail the design of the matrix product operator, which was complex due to the long range interactions present in this model, and was my primary contribution to the article. Finally we will present, and comment on results from this investigation.

\section{Introduction}
In a perfect crystal the wave functions are described by Bloch states, which are delocalized over the entire crystal, thus allowing transport. On the other hand, an imperfect crystal where disorder or impurities create scattering centres leads to localized wave functions, a phenomenon known as Anderson localization \cite{Anderson58,LR85,SSC13}. In fact the key to localization is destructive interference of the wave functions and this can also be induced through geometric constraints on the tunneling rates in the lattice. Such systems allow the construction of flat band states with infinite effective mass (zero kinetic energy) which are insulating stationary states of the system. Synthetic flat band crystals have recently been demonstrated in a variety of systems including photonic lattices \cite{GuzmanSilvaEA14,VicencioEA15,MukherjeeEA15,MT15}, polaritons in etched semiconductor heterostructures \cite{JacqminEA14,BabouxEA16}, ultracold atomic gases in optical lattices \cite{TOINNT15}, surface plasmons \cite{NONK12,KUNNK16}, and they have been proposed in superconducting resonators \cite{YangEA16}. As remarked earlier in this thesis, the photonic systems we tend to consider are inherently lossy, and so a coherent or incoherent drive is required to repopulate the system for a non-vacuum stationary state. The aim of this work then, was to investigate the properties of a geometrically frustrated lattice in the driven-dissipative regime.  

\subsection{The Model}
The model is a sawtooth lattice -- consisting of two one-dimensional sublattices, labelled \(A\) and \(B\). Each site in the \(B\) sublattice is coupled to its two adjacent sites with some tunneling rate \(t\). Each site in the \(A\) sublattice is connected to its two adjacent sites in the \(B\) sublattice with some tunneling rate \(t'\), but is \emph{not} connected to other sites in the \(A\) sublattice. Both sublattices have an interaction energy \(U_{X}\) and each site has a coherent drive with amplitude \(\Omega_{X,i}\), where the label \(X\) is \(A\) or \(B\) denoting the sublattice. The dissipative regime is one where the two sublattices have independent dissipation rates \(\gamma_{X}\) where again \(X = A,B\), and each site is dissipatively coupled to its own independent bath. The model is shown diagrammatically in \cref{fig:dim1-1}.

\begin{figure}[ht!]
\centering
\includegraphics[width=0.8\linewidth]{\figpath/DIM_lattice_diagram}
\caption{The lattice with tunneling rates \(t\) and \(t'\) along with the labeling of the \(A\) and \(B\) sublattices and the unit cell \(i\). Excitations dissipate at a rate \(\gamma_{A(B)}\) from the individual sites of the \(A (B)\) sublattice and coherent or incoherent drives are applied with amplitude \(\Omega_{X, i}\) or intensity \(P_{X, i}(X = A,B)\). Reproduced from ref.~\cite{OBH17}.}
\label{fig:dim1-1}
\end{figure}

In the site basis the Hamiltonian for the system is,
\begin{equation}
	H = H_{0} + H_{t} + H_{U} + H_{D},
	\label{eq:dim1-1}
\end{equation}
where,
	\begin{align}
		H_{0} &= \sum_{i} \omega_{0}(\hat{a}_{i}^{\dagger}\hat{a}_{i} + \hat{b}_{i}^{\dagger}\hat{b}_{i}), \label{eq:dim1-2} \\
		H_{t} &= \sum_{i} \left[ t(\hat{b}_{i-1}^{\dagger}\hat{b}_{i} + \hat{b}_{i-1}\hat{b}_{i}^{\dagger}) + t'(\hat{b}_{i}^{\dagger}\hat{a}_{i} + \hat{b}_{i}^{\dagger}\hat{a}_{i-1} + \hat{b}_{i}\hat{a}_{i-1}^{\dagger} + \hat{b}_{i}\hat{a}_{i}^{\dagger})\right], \label{eq:dim1-3} \\
		H_{U} &= \sum_{i} \left[ U_{A}\hat{a}_{i}^{\dagger}\hat{a}_{i}^{\dagger}\hat{a}_{i}\hat{a}_{i} + U_{B}\hat{b}_{i}^{\dagger}\hat{b}_{i}^{\dagger}\hat{b}_{i}\hat{b}_{i}\right], \label{eq:dim1-4}
	\end{align}
	where \(\omega_{0}\) is some on-site energy (setting \(\hbar=1\) and working in terms of frequencies, as is common in this field), and \(\hat{a}_{i} (\hat{b}_{i})\) is the bosonic annihilation operator for the site \(i\) on sublattice \(A(B)\). We defer specification of the driving Hamiltonian, \(H_{D}\), until after we have introduced the Wannier basis. The master equation then has the standard Lindblad form,
	\begin{align}
		\dot{\rho} = -i[H, \rho] &+ \frac{\gamma_{A}}{2}\sum_{i}\left[ 2\hat{a}_{i}\rho\hat{a}_{i}^{\dagger} - \{\hat{a}_{i}^{\dagger}\hat{a}_{i}, \rho\} \right] \notag \\
		&+ \frac{\gamma_{B}}{2} \sum_{i} \left[ 2\hat{b}_{i}\rho\hat{b}_{i}^{\dagger} - \{\hat{b}_{i}^{\dagger}\hat{b}_{i}, \rho\} \right].
		\label{eq:dim1-5}
	\end{align}

In the non-interacting regime \((U_{A} = U_{B} = 0)\), the undriven Hamiltonian given by \(H_{0} + H_{t}\) can be written in terms of decoupled Bloch modes with frequencies given by,
\begin{equation}
	E_{k} = \omega_{0} + t \cos k \pm \sqrt{t^{2}\cos^{2}k + 2t^{'2}(1+\cos k)}.
	\label{eq:dim1-6}
\end{equation}
If we take the limit \(t' \rightarrow \sqrt{2}t\) we find,
\begin{align}
	E_{k} &= \omega_{0} + t \cos k \pm \sqrt{t^{2}\cos^{2}k + 2(\sqrt{2}t)^{2}(1+\cos k)}, \notag \\
	&= \omega_{0} + t \cos k \pm \sqrt{t^{2}\cos^{2}k + 4t^{2} + 4t^{2}\cos k}, \notag \\
	&= \omega_{0} + t \cos k \pm \sqrt{t^{2}(\cos^{2}k + 4 + 4\cos k}, \notag \\
	&= \omega_{0} + t \cos k \pm \sqrt{t^{2}(\cos k + 2)^{2}}, \notag \\
	&= \omega_{0} + t \cos k \pm (t\cos k + 2t),
	\label{eq:dim1-7}
\end{align}
which yields a flat lower band at \(E_{0} = \omega_{0} - 2t\) and a \(2t\) gap to the dispersive band, \(E_{k} = \omega_{0} + 2t + 2t\cos k\). This is the geometrically frustrated regime, in which the kinetic energy of the flat band is quenched. We can most easily represent (and investigate) the system in this frustrated state using Wannier states.

\subsection{Wannier Basis} 
The Wannier basis is an orthogonal set of states defined by the summation of bloch states for individual bands in a lattice \cite{Wannier37}. In our case the tight-binding bosonic operators can be expressed as,
\begin{align}
	\hat{a}_{i}^{\dagger} &= \sum_{j} w_{A}(r_{i} - r_{j}) W_{j}^{\dagger}, \label{eq:dim1-8} \\
	\hat{b}_{i}^{\dagger} &= \sum_{j} w_{B}(r_{i} - r_{j}) W_{j}^{\dagger}, \label{eq:dim1-9}  
\end{align}
where,
\begin{align}
	w_{A}(r) &= \frac{\sqrt{2}}{2\pi} \int^{\pi}_{-\pi} \frac{\cos\left(\frac{k}{2}\right)\mathrm{e}^{-ikr}\mathrm{e}^{ik/2}}{\sqrt{\cos(k) + 2}} \mathrm{d}k, \label{eq:dim1-10} \\
	w_{B}(r) &= \frac{-1}{2\pi} \int^{\pi}_{-\pi} \frac{\mathrm{e}^{-ikr}}{\sqrt{\cos(k) + 2}} \mathrm{d}k, \label{eq:dim1-11}
\end{align}
and \(W_{j}\) annihilates an excitation in the flat-band Wannier basis which is exponentially localized on the unit cell \(j\).

For a sufficiently large band gap, \(2t \gg g_{k,i}^{a},g_{k_,i}^{b},w_{A},w_{B}\), where \(g_{k,i}^{a(b)}\) is the coupling strength between the A(B) sublattice and the environment, we can neglect the dispersive band. We may then write our master equation, \cref{eq:dim1-5}, as 
\begin{equation}
	\dot{\rho} = -i[H_{W}, \rho] + \frac{1}{2} \sum_{j,l} \gamma_{l} \left[ 2W_{j} \rho W_{j+l}^{\dagger} - \left\{W_{j+l}^{\dagger}W_{j}, \rho\right\}\right],
	\label{eq:dim1-12}
\end{equation}
where the Hamiltonian is now,
\begin{equation}
	H_{W} = H_{0,W} + H_{\Omega,W} + H_{U,W},
	\label{eq:dim1-13}
\end{equation}
and where,
\begin{align}
	H_{0,W} &= \sum_{i} \Delta W_{i}^{\dagger}W_{i}, \label{eq:dim1-14} \\
	H_{\Omega,W} &= \sum_{i} \left[ \Omega_{W,i}W_{i} + \Omega_{W,i}^{*} W_{i}^{\dagger} \right], \label{eq:dim1-15}
\end{align}
where we have again moved into the rotating frame, such that the detuning of the drive from the flat band \(\Delta = \omega_{0} - 2t - \omega_{D}\), and neglected rapidly rotating terms in \(H_{\Omega}\). We set the coherent drive amplitude such that,
\begin{equation}
	\Omega_{W,i}W_{i} = \sum_{j} \left[ \Omega_{A,j}w_{A}(r_{i} - r_{j}) + \Omega_{B,j}w_{B}(r_{i} - r_{j}) \right].
	\label{eq:dim1-16}
\end{equation}
The interaction term \(H_{U,W}\) is given by substituting \cref{eq:dim1-8,eq:dim1-9,eq:dim1-10,eq:dim1-11} in to \cref{eq:dim1-4} yielding,
\begin{equation}
	H_{U,W} = \sum_{i} \sum_{j',l',m'} \left( U_{j',l',m'}^{\mathrm{eff},A} + U_{j',l',m'}^{\mathrm{eff},B} \right)W_{i}^{\dagger}W_{i+j'}^{\dagger}W_{i+l'}W_{i+m'},
	\label{eq:dim1-17}
\end{equation}
where,
\begin{align}
	U_{j',l',m'}^{\mathrm{eff},A} &= \frac{4U_{A}}{(2\pi)^{3}} \int^{\pi}_{-\pi} \int^{\pi}_{-\pi} \int^{\pi}_{-\pi} \mathrm{d}k' \mathrm{d}q \mathrm{d}q' \frac{\mathrm{e}^{ik'j'}\mathrm{e}^{iql'}\mathrm{e}^{iq'm'}}{\mathcal{N}(k',q,q')} \Pi\left( \frac{k'+q+q'}{2\pi} \right) \notag \\
	&\quad \times \cos\left[(k'+q+q')/2\right]\cos(k'/2)\cos(q/2)\cos(q'/2), \label{eq:dim1-18} \\
	U_{j',l',m'}^{\mathrm{eff},B} &= \frac{U_{B}}{(2\pi)^{3}} \int^{\pi}_{-\pi} \int^{\pi}_{-\pi} \int^{\pi}_{-\pi} \mathrm{d}k' \mathrm{d}q \mathrm{d}q' \frac{\mathrm{e}^{ik'j'}\mathrm{e}^{iql'}\mathrm{e}^{iq'm'}}{\mathcal{N}(k',q,q')} \Pi\left( \frac{k'+q+q'}{2\pi} \right) \label{eq:dim1-19}
\end{align}
and where,
\begin{equation}
	\mathcal{N}(k',q,q') = \sqrt{\left[ \cos(k'+q+q') + 2\right](\cos k'+2)(\cos q+2)(\cos q'+2)},
	\label{eq:dim1-20}
\end{equation}
and \(\Pi(x)\) is the rectangle function.

The Wannier basis dissipation coefficient is given by,
\begin{equation}
	\gamma_{l} = \sum_{i} \sum_{x=A,B} \gamma_{x}w_{x}(r_{i})w_{x}(r_{i} - r_{l}),
	\label{eq:dim1-21}
\end{equation}
which can be simplified to,
\begin{align}
	\frac{\gamma_{l}}{\gamma_{A}} &= (2f_{0} + f_{1}) - (1 - \kappa)f_{0}, &\text{ for }l=0, \label{eq:dim1-22} \\
	\frac{\gamma_{l}}{\gamma_{A}} &= -(1-\kappa)f_{l}, &\text{ for }l \neq 0, \label{eq:dim1-23}
\end{align}
where \(f_{l} = (\sqrt{3}-2)^{|l|}/\sqrt{3}\), \(f_{l}=f_{-l}\) and \(\kappa = \gamma_{B}/\gamma_{A}\), so that the dissipation is entirely local when \(\kappa = 1\). 

Having defined our model in then Wannier basis, we will first consider results from the non-interacting regime, where \(U_{A} = U_{B} = 0\).

\section{Non-interacting Regime}
In this regime the master equation \cref{eq:dim1-12} can be solved using Ehrenfest's theorem \cite{Ehrenfest27,BP_Ehrenfest},
\begin{align}
	\frac{\mathrm{d}}{\mathrm{d}t} \langle \hat{O} \rangle &= \langle \hat{O} \dot{\rho} \rangle, \notag \\
	&= \mathrm{Tr}\left\{ -i\hat{O}[H_{W}, \rho] + \hat{O}\mathcal{D}(W) \right\},
	\label{eq:dim2-1}
\end{align}
where \(\hat{O}\) is the operator for some observable of the system, and \(\mathcal{D}(x)=(\gamma/2)\sum_{j}2x_{j}\rho x_{j}^{\dagger} - \{x_{j}^{\dagger}x_{j}, \rho\}\) is the Lindblad-form dissipator. From Ehrenfest's theorem we find the following equations of motion,
\begin{align}
	\frac{\mathrm{d}}{\mathrm{d}t} \langle W_{i} \rangle &= -\frac{i}{2}\Omega_{W}^{*}\delta_{i,0} - \sum_{l} \gamma_{l}\langle W_{i+l} \rangle, \label{eq:dim2-2} \\
	\frac{\mathrm{d}}{\mathrm{d}t} \langle W_{i}^{\dagger}W_{i+j} \rangle &= \frac{i}{2} \langle \Omega_{W}\delta_{i,0} W_{i+j} - \Omega_{W}^{*}\delta_{i+j,0}W_{i}^{\dagger} \rangle \notag \\ 
	&\quad - \frac{1}{2}\sum_{l} \gamma_{l} (\langle W_{i+l}^{\dagger}W_{i+j} \rangle + \langle W_{i}^{\dagger}W_{i+j-l}\rangle), \label{eq:dim2-3}
\end{align}
where the Kronecker delta, \(\delta_{i,0}\), in the drive term indicates that the system is driven only in the Wannier state localized about site zero. \Cref{eq:dim2-2,eq:dim2-3} provide a set of coupled equations which we solve numerically. A cutoff is applied to both the nonlocal dissipation coefficient \(\gamma_{l}\), which decays exponentially with \(l\), and to the correlations which we expect to be negligible at long-range. As such we make the approximations that \(\gamma_{l} = 0\) for \(l > 10\), and that \(\langle W_{i}^{\dagger}W_{i+j} \rangle = 0\) for \(j > 10\).

\subsection{Dissipation-induced mobility}

\begin{figure}[ht!]
\centering
\subfloat{\includegraphics[width=0.32\linewidth]{\figpath/log_Wannier_dens}}%
\subfloat{\includegraphics[width=0.32\linewidth]{\figpath/b_site_density}}%
\subfloat{\includegraphics[width=0.32\linewidth]{\figpath/log_Wannier_dens_incoherent}}
\caption{Results from finding the stationary state in the non-interacting regime. (a) Normalized excitation density in the Wannier basis for coherent drive with amplitude \(\Omega_{W} = \gamma_{A}\) at the \(i=0\) Wannier state. (b) Normalized density of the \(B\) sublattice. (c) Normalized density with an incoherent pump with intensity \(P_{B,i=0} = \gamma_{A}/100\).}
\label{fig:dim2-1}
\end{figure}

In \cref{fig:dim2-1} we see results from solving \cref{eq:dim2-2,eq:dim2-3}. \Cref{fig:dim2-1}~(a) shows the density of Wannier state excitations, normalized by the density on the pumped site, \(i=0\). It can be seen that in spite of the lack of coherent transport processes, in the steady state the density is found to be non-zero away from the pumped site. This effect is also shown in \cref{fig:dim2-1}~(b) which shows the \(B\) sublattice density \(\langle \hat{b}_{i}^{\dagger}\hat{b}_{i} \rangle\). Recall that the Wannier drive amplitude is given by \cref{eq:dim1-16}, so in the lattice site basis the drive amplitudes \(\Omega_{A,i}\) and \(\Omega_{B,i}\) are non-zero for \(i \neq 0\), however the density profile widens noticeably as \(\kappa\) tends to \(0\), increasing the amount of non-local dissipation, indicating that the effect is not simply an artifact of transformation into the Wannier basis.

\begin{figure}[ht!]
\centering 
\includegraphics[width=0.8\linewidth]{\figpath/xi_vs_kappa}
\caption{\label{fig:dim2-2}Decay length of the density profile \(\xi_{i} = |\log_{10}N_{i} - \log_{10}N_{i+1}|^{-1}\) for \(i=4\) as a function of \(\kappa = \gamma_{B}/\gamma_{A}\). The solid line gives the results of the exact model, the dashed line of the effective drive model, and the dot-dashed line of the direct coupling model. In all three models the Wannier state \(W_{0}\) was driven with amplitude \(\Omega_{W} = \gamma_{A}\).}
\end{figure}

In the Wannier basis the density decays exponentially from the driven site, and can be approximated by,
\begin{equation}
	N_{i} = \mathrm{e}^{-\frac{|r_{i}|}{\xi}},
	\label{eq:dim2-4}
\end{equation}
in turn allowing one to extract a decay length \(\xi_{i} = |\log_{10}N_{i} - \log_{10}N_{i+1}|^{-1}\). This decay length is shown in \cref{fig:dim2-2} for \(i=4\). Recalling \cref{eq:dim1-22,eq:dim1-23}, as \(\kappa\) decreases, the nonlocal dissipation rates \(\gamma_{l}\) increase, leading to a divergence in the decay length as \(\kappa \rightarrow 0\). In this limit, the \(B\) sublattice dissipation rate tends to zero and the dark state \(\sum_{i} (-1)^{i}W_{i}^{\dagger}|0\rangle = \sum_{i}(-1)^{i}\hat{b}_{i}^{\dagger}|0\rangle\) forms. This dark states extends over the entire lattice, hence the divergence in \(\xi\). The limit \(\kappa \rightarrow \infty\), where the \(A\) sublattice dissipation rate tends to zero, is inaccessible with the driving mechanism we use, as the corresponding dark state \(\sum_{i}(-1)^{i}\hat{a}_{i}^{\dagger}|0\rangle\) would involve contributions from the dispersive band.  

\subsection{Long-range first-order coherences}
In the coherently driven system the first-order coherence,
\begin{equation}
	g^{(1)}(j,l) = \frac{\langle W_{j}^{\dagger}W_{l} \rangle}{\sqrt{\langle W_{j}^{\dagger}W_{j} \rangle \langle W_{l}^{\dagger}W_{l} \rangle}},
	\label{eq:dim2-5}
\end{equation}
is perfectly correlated with \(g^{(1)}(j,l) = (-1)^{|j-l|}\) as the density matrix is a product of coherent states with alternating phases. Long-range first-order coherence therefore exists in spite of the exponentially decaying density profile. This is indicative of a condensate whose extension is the decay length \(\xi\) of the spatial density profile. 

In order to check that the coherence we find is not simply inherited from the coherent drive, we replace the coherent drive in the Wannier basis (setting \(\Omega_{W} = 0\)) with an incoherent pump in the site basis of the form,
\begin{equation}
	\mathcal{L}_{\mathrm{pump}} = \sum_{i}\left[\frac{P_{A,i}}{2}\left(2\hat{a}_{i}^{\dagger}\rho\hat{a}_{i} - \{\hat{a}_{i}\hat{a}_{i}^{\dagger}, \rho\}\right) + \frac{P_{B,i}}{2}\left(2\hat{b}_{i}^{\dagger}\rho\hat{b}_{i} - \{\hat{b}_{i}\hat{b}_{i}^{\dagger}, \rho\}\right) \right],
	\label{eq:dim2-6}
\end{equation}
where \(P_{x,i}\) is the intensity of pump on the \(i^{\mathrm{th}}\) site of the \(x = A,B\) sublattice. In fact we pump only the \(B\) sublattice and, as with the coherent drive, we pump only one site, \(i=0\). This driving scheme leads to the density profile shown in \cref{fig:dim2-1}~(c), which also shows exponential decay. The resulting first-order coherences are shown in \cref{fig:dim2-3}, and we see that the dissipation-induced mobility continues to generate significant coherences, even without coherent input. In the Wannier basis the incoherent drive is still localised around site zero, and so suppresses coherence around \(g^{(1)}(0,l)\). 

\begin{figure}[ht!]
\centering 
\includegraphics[width=0.8\linewidth]{\figpath/g1_incoherent_final}
\caption{\label{fig:dim2-3}Spatial coherences in the steady state \(g^{(1)}(i,j)\) for an incoherent pump at \(i=0\) with \(P_{B,0} = \gamma_{A}/100\) and \(\kappa = 0.1\).}
\end{figure}

\subsection{Modeling the decay length of the dissipation-induced mobility}
\begin{figure}[ht!]
\centering 
\includegraphics[width=0.8\linewidth]{\figpath/DIM_mobility_models}
\caption{\label{fig:dim2-4}Diagrammatic representations of the models we consider to explain dissipation-induced mobility in the non-interacting regime. The central site is pumped with a drive of strength \(\Omega_{W,0}\) and there is a dissipation rate \(\gamma_{0}\) on each site. (a) The single-site diffusion model assumes that only tunneling between neighbouring sites is important. (b) The direct nonlocal dissipative coupling model, where each site is coupled directly to the pumped site (site 0) by a nonlocal dissipation term. (c) The effective drive model consists of a three-site model where the pumped site is not affected by the two remote, coupled, unpumped sites. Reproduced from ref.~\cite{OBH17}.}
\end{figure}
In order to clarify that the transport we observe \emph{is} the result of non-local dissipation processes, we first attempt to model it as a diffusion process, and compare the decay length of the resulting density profile. This diffusive random walk model, shown diagramatically in \cref{fig:dim2-4}~(a), has the following continuity equation,
\begin{equation}
	\frac{\partial N_{i}}{\partial t} = - \left.\frac{\partial N_{i}}{\partial t}\right|_{L} - \left.\frac{\partial N_{i}}{\partial t}\right|_{R} - \Gamma N_{i} + F(x_{i}),
	\label{eq:dim2-7}
\end{equation}
where \(N_{i}\) is the excitation density on the site \(i\), \(\partial N_{i} / \partial t|_{L,R}\) is the number of excitations on site \(i\) moving to the left (\(L\)) and right (\(R\)), \(\Gamma\) is the dissipation rate, and \(F(x_{i})\) is a source distribution. The net particle flow to the right, from the site \(i\) to the site \(i+1\), in a time interval \(\Delta \tau\) is given by,
\begin{equation}
	\left.\frac{\partial N_{i}}{\partial t}\right|_{R} = -J\frac{\Delta x}{\Delta \tau}\frac{\partial N_{i}}{\partial x},
	\label{eq:dim2-8}
\end{equation}
where \(J\) is the hopping rate between adjacent sites. The same equation holds for \(\partial N_{i}/\partial t|_{L}\). The diffusion equation is then,
\begin{equation}
	\frac{\partial N_{i}}{\partial t} = J\frac{\Delta x^{2}}{\Delta t}\frac{\partial^{2} N_{i}}{\partial x^{2}} - \Gamma N_{i} + F(x_{i}),	
	\label{eq:dim2-9}
\end{equation}
where \(\Delta x^{2}/\Delta \tau\) is a scaled diffusion constant \(D\). Without sources, the steady-state solution of \cref{eq:dim2-9} is,
\begin{equation}
	N_{i} = A\mathrm{e}^{-\frac{x_{i}}{\Xi}} + B\mathrm{e}^{\frac{x_{i}}{\Xi}},
	\label{eq:dim2-10}
\end{equation}
where \(\Xi = \sqrt{JD/\Gamma}\) is the decay length.

If the dissipative coupling can be modelled as a diffusive process, we should be able to recover the same functional form for the decay length. Assuming \(\Gamma \propto \gamma_{0}\) and \(J \propto \gamma_{1}\) (the \(i+1\) nonlocal dissipation constant) we find,
\begin{equation}
	\Xi(\kappa) \propto \sqrt{\frac{\gamma_{1}}{\gamma_{0}}} \propto \left(\frac{2f_{0} + f_{1}}{f_{0}(1-\kappa)} - 1\right)^{-\frac{1}{2}},
	\label{eq:dim2-11}
\end{equation}
which is plotted in \cref{fig:dim2-5}, and clearly has a different form to the decay length extracted from the exact calculation shown in \cref{fig:dim2-2}. Even if we include the direct nonlocal dissipative coupling of the pump site to remote sites as a source term in \cref{eq:dim2-9}, we find the same functional form for \(\Xi(\kappa)\). From this, we conclude that the transport we see cannot be explained by diffusion processes.

\begin{figure}[ht!]
\centering 
\includegraphics[width=0.8\linewidth]{\figpath/DIM_DiffDecay}
\caption{\label{fig:dim2-5}Plot of the diffusive decay constant, \(\Xi(\kappa) \propto \sqrt{\gamma_{1}/\gamma_{0}}\) against \(\kappa\).}
\end{figure}

We next consider which of the nonlocal dissipative processes contribute significantly to the mobility by comparing the predicted density distributions for two approximate models to that we find by numerically solving the exact model. First we consider an approximation in which we neglect the dissipative coupling between sites which are not pumped. Transport between the pumped site \(i=0\) to any other site, \(j\), is therefore mediated solely by the nonlocal dissipative constant \(\gamma_{j}\), a model which is shown diagramatically in \cref{fig:dim2-4}~(b). 

The density distribution of this model can be calculated by solving the Ehrenfest equations \cref{eq:dim2-2,eq:dim2-3} for a two-site system consisting of site zero and site \(j\), where site zero is pumped with a drive strength \(\Omega_{W}\). Doing so yields,
\begin{align}
	\langle W_{j}^{\dagger}W_{j} \rangle &= \frac{\gamma_{j}^{2}|\Omega_{W}|^{2}}{4(\gamma_{0}^{2} - \gamma_{j}^{2})^{2}}, \label{eq:dim2-12} \\
	&\approx \frac{\gamma_{j}^{2}|\Omega_{W}|^{2}}{4\gamma_{0}^{4}}, \label{eq:dim2-13}
\end{align}
where we justify the approximation \(\gamma_{0} \gg \gamma_{j}\) by recalling that the Wannier states are exponentially localised. The decay length in this model is then given by,
\begin{equation}
	\xi_{\Omega} \approx \left[ \log_{10}\left(\frac{\gamma_{j+1}^{2}}{\gamma_{j}^{2}}\right)\right]^{-1} \approx 0.38,
	\label{eq:dim2-14}
\end{equation}
which is plotted as the dot-dashed line in \cref{fig:dim2-2}. It can be seen that in the limit \(\kappa \rightarrow 1\) the nonlocal dissipation rates vanish, and this model predicts the density distribution well.

As an improved approximation we consider a three-site model with nearest-neighbour hopping, and sites 0, \(j\), and \(j+1\), where \(j > 1\) so that the \(j,j+1\) sites are not directly coupled to the pumped site. This model is shown diagramatically in \cref{fig:dim2-4}~(c). We are again solving the Ehrenfest equations of motion, and make the assumption that for \(i=\{j,j+1\}\) only the contribution from the pumped site is relevant in \cref{eq:dim2-2}, such that,
\begin{equation}
	\sum_{l \neq \{0,1\}} \gamma_{l} \langle W_{j+l} \rangle \approx \gamma_{-j} \langle W_{0} \rangle.
	\label{eq:dim2-15}
\end{equation}
We also assume that the two unpumped sites \(j, j+1\) do not affect the field amplitude of site zero. This means that it is a constant determined by evaluating the three-site system \(i=\{-1,0,1\}\). Doing so, we find that
\begin{equation}
	\langle W_{0} \rangle \approx \frac{-i\gamma_{0}\Omega_{W}^{*}}{2(\gamma_{0}^{2} - 2\gamma_{1}^{2}},
	\label{eq:dim2-16}
\end{equation}
which we can substitute back in to \cref{eq:dim2-2} in order to solve the the \(i=\{0,j,j+1\}\) model. We find that the pumped site acts as an effective drive for the two-site system with coefficients,
\begin{align}
	\Omega_{W,j}^{*} &= \frac{\gamma_{0}\gamma_{j}\Omega_{W}^{*}}{\gamma_{0}^{2} - 2\gamma_{1}^{2}}, \label{eq:dim2-17} \\
	\Omega_{W,j+1}^{*} &= \frac{\gamma_{0}\gamma_{j+1}\Omega_{W}^{*}}{\gamma_{0}^{2} - 2\gamma_{1}^{2}}, \label{eq:dim2-18}
\end{align}
which we this time substitute in to \cref{eq:dim2-3} to determine the approximate density profile of this model. From this we calculate the approximate decay length,
\begin{equation}
	\xi \approx \left[ 2\log_{10}\left(\frac{\gamma_{0}\gamma_{j} - \gamma_{1}\gamma_{j+1}}{\gamma_{1}\gamma_{j} - \gamma_{0}\gamma_{j+1}}\right)\right]^{-1},
	\label{eq:dim2-19}
\end{equation}
which is plotted as the dashed line in \cref{fig:dim2-2}. This approximation works well for a wide range of nonlocal dissipation strengths, but breaks down as \(\kappa \rightarrow 0\). This is due to multiple hopping processes not being included in the model. 

We conclude that for \(\kappa \approx 1\), where the Wannier states are decoupled, direct dissipative coupling between the pumped site and each other site plays the most significant role. As \(\kappa\) decreases transfer of particles from neighbouring sites plays an increasing role, and as \(\kappa\) decreases further the extent of these significant interactions increases to next-nearest neighbour and beyond. 

\section{Strongly Interacting Regime}
We next consider the interacting regime, which in the Wannier basis results in nonlocal terms, like those in the dissipator. The exact form of the interaction term in the Wannier basis Hamiltonian is given in \cref{eq:dim1-17,eq:dim1-18,eq:dim1-19,eq:dim1-20}. Such nonlocal terms can also result in transport, a phenomenon which has been explored in detail \cite{HA10,TVNH13,TKWA13,PCOV15,PM15}. This is suppressed in the limit of very strong interaction, and so it is this limit we explore in order to explore the effect of the dissipation. The very strong interaction limit is also advantageous as it allows us to truncate to the single excitation subspace. To facilitate this truncation we calculated the second-order correlation function for the pumped site \(\langle W_{0}^{\dagger}W_{0}^{\dagger}W_{0}W_{0} \rangle\) under the assumption that the pumped site is completely decoupled. This gives an upper bound on the probability that the pumped site, which will always have the highest density, has at least two excitations. This correlation function can be calculated exactly using an approach by Drummond and Walls \cite{DW80,BOC13}, the result of which is plotted in \cref{fig:dim3-1} as a function of \(U_{0}/\Omega_{W,0}\) and \(\gamma_{0}/\Omega_{W,0}\) where \(U_{0} = U_{0,0,0}^{\mathrm{eff},A} + U_{0,0,0}^{\mathrm{eff},B}\). This allows us to select a regime where the double-excitation probability on site zero is much lower than the density on the adjacent sites, \(\langle W_{0}^{\dagger}W_{0}^{\dagger}W_{0}W_{0} \rangle < \langle W_{1}^{\dagger}W_{1} \rangle\), validating the truncation to the single-excitation subspace.

\begin{figure}[ht!]
\centering 
\includegraphics[width=0.8\linewidth]{\figpath/doubleocc_prob}
\caption{\label{fig:dim3-1}Second-order correlation function for the pumped site \(\log(\langle W_{0}^{\dagger}W_{0}^{\dagger}W_{0}W_{0}\rangle )\) with coherent and incoherent transfer processes set to zero.}
\end{figure}

\subsection{MPO Design}