In this chapter we will consider some of the theoretical background of many-body and open quantum systems. These are rather broad fields, so we shall focus in particular on aspects that are relevant to the research presented later in this thesis. We will begin by considering what is meant by the word `many' in many-body quantum systems, and why this presents such a challenge to physicists. We will introduce the driven-dissipative Bose-Hubbard model, and physically motivate its terms in the photonic setting, in order to make later chapters more transparent. Then we will discuss what is meant by `open' quantum systems, and discuss how the dynamics of such a system are calculated. Finally we shall discuss how to reformulate these dynamics as a system of coupled linear equations.  

\section{Many-Body Quantum Physics}

\subsection{How many?}
Quantum physics has a problem. More than one actually, but the reader may safely assume that the author considers musing on the interpretation of quantum mechanics to be far above his pay-grade, and that he belongs to the ``Shut up and calculate!'' school of thought \cite{Mermin89}. What hardship then, does quantum mechanics present to those of us interested only in crunching numbers and getting results? 

To answer that, let us first step back and consider a classical system. Consider some system of \(N\) components, each of which can be in one of two possible states. In total there are \(2^{N}\) possible configurations of the system, each of which can be \emph{completely} described by an \(N\)-bit string. Furthermore, if we increase \(N\) to \(N+m\), we only need to add \(m\) bits. Even if we change to a system where there are three possible states of each component, although the total state space increases in size to \(3^{N}\), since the system must exist in \emph{one and only one} of those configurations, we can still efficiently represent the system with just \(N\) bits. Mathematical representations of many-body classical systems scale linearly with the size of the system (the number of `bodies'). This does not mean that classical many-body problems are \emph{easy}, just that they get harder only in proportion with the size of the system.  

Enter quantum mechanics. We may again consider a system of \(N\) components, each of which can be in one of two possible states, meaning there are \(2^{N}\) possible configurations -- so far, so good. However, there is a fundamental principle of quantum mechanics -- the Superposition principle -- that says that if a system may be in one of two states, which we shall label \(|0 \rangle\) and \(|1\rangle\), then it may also be in the state,
\begin{equation}
	| \psi \rangle = \alpha |0\rangle + \beta |1\rangle,
	\label{eq:mbq1-1}
\end{equation}
where \(\alpha\) and \(\beta\) are complex coefficients. This means that we must replace each of our bits with a complex vector,
\begin{equation}
	|\psi \rangle = \begin{pmatrix}
						\alpha \\
						\beta
					\end{pmatrix},
	\label{eq:mbq1-2}
\end{equation}
where we have arbitrarily chosen a convention that the first element corresponds to \(|0\rangle\), and the second to \(|1\rangle\). Nevertheless, if all that is required to transition to the quantum regime is to replace \(N\) integers to \(2N\) complex floats, this is not so bad. Unfortunately, the superposition principle applies equally to the composite system. Taking \(N=3\), if \(|\Psi\rangle = |000\rangle\) and \(|\Psi\rangle = |111\rangle\) are valid configurations, so is 
\begin{equation}  
	| \Psi \rangle = c_{000}|000 \rangle + c_{111}| 111 \rangle,
	\label{eq:mbq1-3}
\end{equation}
where \(c_{000}\) and \(c_{111}\) are again complex coefficients. In fact, any arbitrary combination of the \(2^{3} = 8\) possible states of the system,
\begin{equation}
	| \Psi \rangle = \sum^{1}_{i,j,k=0} c_{ijk} |ijk \rangle,
	\label{eq:mbq1-4}
\end{equation}
is a valid state, so we must use a vector of 8 complex values to describe the state. More generally, if we have some system of \(N\) quantum components, each of which may be in any combination of \(d\) local states, we require a state vector of \(d^{N}\) complex elements. The representation of the system grows \emph{exponentially} with its size. This scaling problem is the crux of many-body quantum physics \cite{Barnett_MS,NielsenChuang_CS}. To compound the issue, it is also clear that the properties of a many-body system are unlikely to be well predicted by single- or even few-body systems -- it has long been understood that in nature ``more is different'' \cite{Anderson72}.

\Cref{tab:mbq1-1} provides some examples of how different objects scale in a system with \(d\) local states, and \(N\) sites. The Hamiltonian is an operator which provides an energy description of a quantum system, and generates unitary (non-dissipative) dynamics, but in fact any operator which acts on the many-body quantum state vector will have the same size. The Liouvillian describes the non-unitary dynamics of a quantum system, and is typically written as a super-operator acting on the density matrix, \(\rho\). In order to solve it numerically we re-write it as a system of linear equations acting on the vectorised density matrix, \(|\rho\rangle\rangle\). Since we are primarily interested in stationary states of dissipative systems -- given by solving \(\hat{L}|\rho_{\mathrm{SS}}\rangle\rangle = 0\) -- it is this \(d^{4N}\) element matrix which concerns us the most. It is precisely this very poor exponential scaling that the Matrix Product State technique, described in detail in the next chapter, was designed to defeat. 

\begin{table}[hb!]
	\centering
	\begin{tabu}{lcc}
		\hline
		Object & Symbol & Size \\
		\hline
		State Vector & \(|\Psi\rangle\) & \(d^{N} \times 1\) \\
		Hamiiltonian & \(\hat{H}\) & \(d^{N} \times d^{N}\) \\
		Density Matrix & \(\rho\) & \(d^{N} \times d^{N}\) \\
		Vectorised Density Matrix & \(|\rho \rangle\rangle\) & \(d^{2N} \times 1\) \\
		Liouvillian Matrix & \(\hat{L}\) & \(d^{2N} \times d^{2N}\)
	\end{tabu}
	\caption{\label{tab:mbq1-1}}
\end{table}

\FloatBarrier
\subsection{Strongly Interacting Light}
The research presented in this thesis deals exclusively with photonic systems in the strongly interacting regime, however the origins of photonic interaction are not immediately obvious. Photons follow boson statistics, so are inherently non-interacting and furthermore, as they travel at the speed of light, opportunities for interaction are limited. In spite of this, systems can be engineered to trap light and then mediate strong interaction between photons. We will consider one such system as an example -- non-linear cavity arrays. We will discuss how they can be engineered to fit the driven-dissipative Bose-Hubbard type model we mostly consider in the research presented in this thesis. 

\subsubsection{Non-Linear Cavity Array}
We begin by stating the model we intend to show is physically realisable,
\begin{equation}
	\mathcal{H} = \sum_{j} \left[\mu\hat{a}_{j}^{\dagger}\hat{a}_{j} + \frac{U}{2}\hat{a}_{j}^{\dagger 2}\hat{a}_{j}^{2} + J(\hat{a}_{j}\hat{a}_{j+1}^{\dagger} + \hat{a}_{j}^{\dagger}\hat{a}_{j+1}) + \frac{\Omega}{\sqrt{2}}\hat{a}_{j}^{\dagger} + \frac{\Omega^{*}}{\sqrt{2}}\hat{a}_{j} \right],
	\label{eq:mbq2-1}
\end{equation}
where \(\hat{a}_{j} (\hat{a}_{j}^{\dagger})\) is the bosonic annihilation (creation) operator on the lattice site \(j\), \(\mu\) is the chemical potential, \(U\) is the interaction energy, \(J\) is the hopping rate between sites, and \(\Omega\) is the coherent drive strength, including a time-dependent phase, however we will typically neglect that by moving in to the rotating frame. In the rotating frame \(\mu\) will be replaced by a detuning of the drive from the energy levels of the system \(\Delta\), which we shall often neglect. Additionally we consider a standard Lindblad-form dissipator. We leave discussion of precisely what that means to the following section, however for now we do intend to show that we can control the dissipation rate \(\gamma\). This is the bosonic form of the fermionic Hubbard model \cite{Hubbard63} from solid-state physics, first considered by Fisher et al. in the context of cold-atom systems \cite{FWGF89,HBP08}. We modify it with the inclusion of a drive and a dissipator, as both are necessary for the generation of non-trivial dynamics -- a bosonic system with drive and no dissipation will trivially fill to infinity, while a bosonic system with dissipation and no drive will be trivially empty. So then, how do we trap light?

\begin{figure}[ht!]
\centering
\includegraphics[width=0.8\linewidth]{\figpath/FPetalon}
\caption{Diagram of a simple planar cavity. Light enters through the mirror \(M_{1}\) and is then reflected between the two mirrors through some medium with a refractive index \(n\). Here we have assumed that \(M_{1}\) transmits perfectly in to the cavity and then light is perfectly reflected by both \(M_{1}\) and \(M_{2}\). Under this approximation the light is confined indefinitely and travels an infinite distance within the cavity. Only light whose wavelength is resonant with the cavity length, \(\lambda = 2nL_{\mathrm{cav}} / m\) where \(m\) is some integer, will survive under these conditions. Over an infinite number of round trips through the cavity even a small amount of destructive interference will result in total elimination of that wavelength of light. Obviously, real systems have losses in the form of absorption, transmission, and scattering, and will therefore neither perfectly isolate a single wavelength (and its harmonics), nor trap light indefinitely.}
\label{fig:mbq2-1}
\end{figure}

An optical cavity is any physical system in which light can be contained. The simplest such system is the Fabry-P\'{e}rot Etalon, a planar cavity consisting of two mirrors with some medium of refractive index \(n\) between \cite{PF1899,Fox_OC}. Such a system is shown diagramatically in \cref{fig:mbq2-1}. In the limit where the mirrors are perfectly reflective on the sides facing in to the cavity, and there are no absorption or scattering losses in the cavity medium, the light will travel an infinite number of round trips through the cavity and will be totally annihilated unless it is precisely resonant with the cavity length. That is,
\begin{equation}
	\lambda = \frac{2nL_{\mathrm{cav}}}{m},
	\label{eq:mbq2-2}
\end{equation}
where \(m\) is some integer. Under these conditions, the light will be trapped indefinitely in the cavity. Obviously, the requirement that there is no absorption or scattering, and perfect reflectivity is physically unrealistic, so it is quite natural to consider such systems in the dissipative regime.

In fact whether the physical origin of the photon loss is absorption or scattering by a mirror or the cavity medium, or by transmission from the cavity is essentially irrelevant to our dynamical equations. All three processes result in a photon leaving the system. In optics the ability of a cavity to retain light is usually referred to as its `quality factor', defined as
\begin{equation}
	\mathcal{Q} = \frac{4\pi L_{\mathrm{cav}}}{\lambda} \frac{1}{-\ln(p)},
	\label{eq:mbq2-3}
\end{equation}
where \(p\) is the fraction of initial power remaining in the cavity after one round trip \cite{Siegman_chp11}. This measure includes loss from all possible sources, and can be exceedingly high in optical microcavities (\(>200\) million \cite{YOLYYV17}). This is important, as it allows us to reasonably neglect these losses \emph{which we cannot control}. Dissipation which we \emph{can} control is introduced through the Purcell effect.

First presented to the American Physical Society by Edward Purcell in 1946 \cite{Purcell46}, the Purcell effect is the phenomenon in which spontaneous emission is enhanced (or suppressed) by the presence of a resonant cavity \cite{Fox_Purcell}. The \emph{Purcell factor} is the ratio of spontaneous emission rate in the presence of a second cavity to the original rate, so a purcell factor greater than one implies an enhancement to the spontaneous emission rate. The Purcell factor is proportional to the Q factor of the cavity, and inversely proportional to its modal volume, so smaller cavities with higher Q factors generate larger Purcell factors. In sum, this means that we can \emph{in principle} envision systems in which we are able to tune the dissipation rate on specific transitions. Naturally fabrication of such a device while possible \cite{GSGLCTM98}, is undoubtedly challenging. Be thankful then, that we are theorists.   

Having discussed how light can be trapped in an optical cavity, and how loss rates can be controlled, we are left with four parameters still to be accounted for -- the drive strength \(\Omega\), chemical potential \(\mu\), hopping rate \(J\), and interaction strength \(U\).

The physical origin of the drive strength is trivial to consider -- the drive strength parameterise the field strength of whatever pump source is used to introduce photons into the system. The chemical potential is a little more complicated. In general it does not exist for photons, as they are bosons, however it is possible to create an effective chemical potential in some systems \cite{HAT15,MOHSS17,LBSRFCC17} and this will be of some importance to our work in Ref.~\cite{BH17}. Finally we note that under the transformation in to a frame that rotates with a coherent drive (i.e. a laser) the chemical potential term is replaced with \(\Delta\), a detuning of the drive from the un-driven system's local eigenstates. 

Hopping between cavities (or sites in our lattice) is again easily understood. Confinement perpendicular to the cavity mode is limited and the wavefunction of the trapped light will extend somewhat beyond the cavity boundaries. Placing the cavities in close enough proximity to one another will result in an overlap between the wavefunctions and consequently tunnelling between sites \cite{HBP06,HBP08}. It is also possible to create stronger couplings between sites by constructing explicit couplings, as has been demonstrated for example in superconducting circuits \cite{MajerEA}.

Finally, we must consider how to introduce and control non-linearity in our cavity array system, and thus define the interaction strength, \(U\). In principle a term of the form \(\hat{a}^{\dagger 2}\hat{a}^{2}\) appears as a result of the optical Kerr effect in certain materials \cite{KY86}, however this effect is third order in the electric susceptibility and therefore requires a high intensity electromagnetic field to become a relevant phenomenon \cite{Boyd_KerrNL}. Given that we typically consider systems with only a few photons this is not really practical. Instead we introduce nonlinearity by considering cavities which contain atoms with a four-level structure. Such systems can produce large Kerr-like nonlinearities, as first shown by Schmidt and Imamo\u{g}lu \cite{SI96}. If we move to considering polaritons (which are photon-atom quasiparticles) then we can construct a full Bose-Hubbard Hamiltonian for our coupled cavity arrays, as detailed in Ref~\cite{HBP08}.

Having discussed the physical origins of an optical driven-dissipative Bose-Hubbard system, we will now consider the theory of open quantum systems.

\section{Open Quantum Systems}

\begin{figure}[ht!]
\centering
\includegraphics[width=0.8\linewidth]{\figpath/open_system}
\caption{The canonical visual representation of an open quantum system. In open quantum systems one considers a composite system, consisting of \(S\), the \emph{system of interest}, here depicted by the blue circle, and \(E\) some well-understood environment -- here depicted by the orange ellipse. The dark blue dashed line marks the interface between the two systems. That the environment is well understood is rather crucial, and often leads to the environment itself being tightly constrained. One does not need to know the state of the environment, but it is necessary to know what states are available, and to be able to precisely define its interaction with the system of interest.}
\label{fig:oqs1-1}
\end{figure}

The study of open quantum systems involves taking a well-behaved closed quantum system, evicting it from the frictionless vacuum every theorist carries in their heart, and embedding it in a noisy environment. The reasons for wanting to do this are obvious -- while the study of energy and number conserving closed quantum systems has lead to many insights, such systems rarely exist. Even a simple pendulum requires consideration of dissipation through friction to describe its real world behaviour.  

In this section we will begin by reminding ourselves of the properties of density matrices, followed by discussion of the Lindblad form master equation, and the approximations inherent to it. Finally, we will discuss some practical details of solving the dynamics of open quantum systems, which will aid in understanding the next chapter on numerical methods.

\subsection{Density Matrix}
The state vector suffices for describing a quantum system which is known to be in one particular state (a \emph{pure} state), but cannot adequately describe a quantum system which may be in one of a number of states. This \emph{ensemble} or \emph{mixed} state must be expressed using the density matrix, commonly denoted by \(\rho\). The density matrix for an ensemble state is defined as,
\begin{equation}
	\rho = \sum_{j} p_{j} |\psi_{j} \rangle \langle \psi_{j}|,
	\label{eq:oqs2-1}
\end{equation}
where \(p_{j} = |c_{j}|^{2}\) is the probability of finding the system in the state \(|\psi_{j}\rangle\) \cite{NielsenChuang_DM}. A valid density matrix has three important properties,
\begin{align}
	\mathrm{Tr}[\rho] &= 1, \label{eq:oqs2-2} \\
	\rho &= \rho^{\dagger}, \label{eq:oqs2-3} \\
	\langle \psi_{j}| \rho | \psi_{j} \rangle &\geq \, 0 \, \forall \, j, \label{eq:oqs2-4}
\end{align}
its trace is equal to one, it is hermitian, and its diagonal values are all greater-than-or-equal to zero (formally it is \emph{positive semidefinite}). All three can be viewed as consequences of the interpretation of the diagonal values of the density matrix as probabilities for each state. The trace condition is simply the requirement that the probability of the system being in \emph{any} state is one, and the other two ensure that the eigenvalues are real and not-negative, making them valid probabilities.

Observables of the system are calculated by taking the trace of the operator product,
\begin{equation}
	\langle \hat{O} \rangle = \mathrm{Tr}[\hat{O}\rho],
	\label{eq:oqs2-5}
\end{equation}
while the system is operated on by applying the operator to `both halves' of the density matrix,
\begin{align}
	\rho (t) &= \hat{U}(t) \rho(0) \hat{U}^{\dagger}(t), \notag \\
	&= \hat{U}(t) | \Psi(0) \rangle \langle \Psi(0) | \hat{U}^{\dagger}(t),
	\label{eq:oqs2-6}
\end{align}  
where we have used a time evolution as an example. Finally, we can calculate the density matrix of a subsystem of a composite system by calculating the \emph{reduced density matrix},
\begin{align}
	\rho^{A} &= \mathrm{Tr}_{B}[\rho^{AB}], \notag \\
	&= \sum_{j,k,l,m} |a_{j} \rangle \langle a_{k}| \langle b_{m} | b_{l} \rangle,
	\label{eq:oqs2-7}
\end{align}
where \(|a_{j} \rangle\) and \(|b_{j} \rangle\) are the basis vectors of the \(A\) and \(B\) subsystems respectively. This property in particular makes the density matrix invaluable to the study of open quantum systems. The system of interest is considered to be part of a composite system with its environment,
\begin{equation}
	\rho = \rho_{S} \otimes \rho_{E},
	\label{eq:oqs2-8}
\end{equation}
where \(\rho_{S}\) is the state of the system of interest, and \(\rho_{E}\) is the state of the environment. The environment can be traced out in order to consider the dynamics of just the system of interest. 

\subsection{Lindblad Master Equation}
Throughout our research we will limit ourselves to consideration of master equations of a particular form -- the Lindblad master equation. We will not derive it here, as this can be found in any good text on open quantum systems, such as Breuer and Petruccione's `The Theory of Open Quantum Systems' \cite{BP_TMQME}. We shall, however, discuss why the Lindblad form is important, and the approximations that are encoded within it. 

The Lindblad master equation, named for G\"{o}ran Lindblad but based on both his work and that of Gorini, Kossakowski, and Sudarshan \cite{Lindblad76,GKS76}, is as follows,
\begin{equation}
	\frac{\mathrm{d}\rho}{\mathrm{d}t} = -i [\mathcal{H}, \rho] + \sum_{j} \frac{\gamma_{j}}{2} \left[ 2\hat{A}_{j} \rho \hat{A}_{j}^{\dagger} - \left\{\hat{A}_{j}^{\dagger}\hat{A}_{j}, \rho\right\}\right],
	\label{eq:oqs3-1}
\end{equation}
where \(\mathcal{H}\) is the Hamiltonian for the system, the operator \(\hat{A}_{j}\) is the Lindblad operator on the site \(j\), which characterise the interaction between the system and its environment, and the anti-commutator \(\{a, b\} = ab + ba\). The first term is in fact just the Liouville-von Neumann equation, which describes the dynamics of the closed (non-dissipative) system \cite{BP_LvN}. The second term, often referred to as the dissipator,
\begin{equation}
	\mathcal{D}[\rho] = \sum_{j}\frac{\gamma_{j}}{2}\left[2\hat{A}_{j}\rho\hat{A}_{j}^{\dagger} - \left\{\hat{A}_{j}^{\dagger}\hat{A}_{j}, \rho\right\}\right],
	\label{eq:oqs3-2}
\end{equation}
encodes the dissipative dynamics. Importantly, a dissipator of this form preserves both the positivity and trace of the density matrix upon which it operates, ensuring physical results. It is quite general, but does implicitly make the following two approximations.

First, the Born approximation. This approximation assumes that the influence of the system of interest on the environment is weak, and that the reduced density matrix for the environment is therefore approximately constant in time. This means that the density matrix for the system-environment composite is,
\begin{equation}
	\rho(t) \approx \rho_{S}(t) \otimes \rho_{E}.
	\label{eq:oqs3-3}
\end{equation} 
This approximation is applied in order to eliminate the state of the environment from the equations of motion, and is valid provided the environment itself is very lossy, such that any excitations decay on a timescale much faster than those in the system of interest. We note that this condition conflicts with the requirements for a strong Purcell enhancement, so we should be wary of trying to model very strong dissipation rates if we are using dissipators tuned to specific transitions. Second, the Markov approximation. This approximation assumes that the system dynamics including the system-environment interactions are dependent only on the current system state, not any prior state. This approximation is made so that it is not necessary to integrate over the entire history of the system in order to determine the current interaction dynamics, and is valid provided that the system itself has no explicit dependency on its history, and again that the decay time of the environment is small compared to the timescale of system dynamics. In this way some excitation which decayed from the system to the environment at some time long ago cannot return to the system. For clarity we note that the approximation is that the time-derivative \(\mathrm{d}\rho(t)/\mathrm{d}t\) has no dependence on \(\rho(\tau \ll t)\), \emph{not} the system \(\rho(t)\) which may very well have some dependence on prior states. These two approximations are often grouped together as the \emph{Born-Markov approximation} \cite{BP_BMS}.

\subsection{The Liouvillian}
The above section explains how we generate the dynamics of the density matrix, but not we find the stationary state,
\begin{equation}
	\frac{\mathrm{d}}{\mathrm{d}t}\rho_{SS} = 0,
	\label{eq:oqs4-1}
\end{equation}
for this we convert the Lindblad into a system of linear equations, which is named the Liouvillian after its classical counterpart. The Lindblad is a \emph{superoperator}, acting on the density matrix from both the left and the right. In order to rephrase the dynamics as a system of coupled linear equations, we shall make use of the following property of matrix products,
\begin{equation}
	 \mathbf{A}\mathbf{X}\mathbf{B} = \left(\mathbf{B}^{T} \otimes \mathbf{A}\right)\bar{X},
	 \label{eq:oqs4-2}
\end{equation}
 where \(\mathbf{A}, \mathbf{B}\) and \(\mathbf{X}\) are matrices, and \(\bar{X}\) is \(\mathbf{X}\) reshaped into a vector \cite{MO13,Roth34}. 
 
 In this way the master equation \cref{eq:oqs3-1} is reformulated as the Liouvillian matrix,
 \begin{align}
 	&\hat{L}|\rho \rangle \rangle = \notag \\ 
 	&\quad \left( \mathbb{I} \otimes -i\mathcal{H} + i\mathcal{H}^{T} \otimes \mathbb{I} + \sum_{j} \frac{\gamma_{j}}{2} \left[ \hat{A}_{j}^{*} \otimes 2\hat{A}_{j} - \mathbb{I} \otimes \hat{A}_{j}^{\dagger}\hat{A}_{j} - \hat{A}_{j}^{\dagger}\hat{A}_{j} \otimes \mathbb{I} \right]\right)|\rho\rangle\rangle,
 	\label{eq:oqs4-3}
 \end{align}
 where \(|\rho \rangle\rangle\) is the vectorised density matrix, and \(\mathbb{I}\) is the identity. Our stationary state is then trivially the solution to the system of equations,
 \begin{equation}
 	\hat{L} |\rho_{SS} \rangle\rangle = 0,
 	\label{eq:oqs4-4}
 \end{equation}
 which for a small enough system we can solve directly using linear algebra methods. For larger systems, we make use of Matrix Product States, which will be the main topic of discussion on our next chapter on numerical methods.