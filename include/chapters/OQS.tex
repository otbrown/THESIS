In this chapter we will consider some of the theoretical background of open quantum systems. This is a rather broad field, so we shall focus in particular on aspects that are particularly relevant to the research presented later in this thesis. We will begin by considering what is meant by an `open' quantum system, then consider what approximations are commonly applied in order to reduce the complexity of the problem. Next we shall consider the three principle effects that open-ness introduces, and finally we shall discuss how to construct a linear system of equations representing the many-body open quantum system.  

\section{Introduction}
\begin{figure}[ht!]
\centering
\includegraphics[width=0.8\linewidth]{\figpath/open_system}
\caption{The canonical visual representation of an open quantum system. In open quantum systems one considers a composite system, consisting of \(S\), the \emph{system of interest}, here depicted by the blue circle, and \(E\) some well-understood environment -- here depicted by the orange ellipse. The dark blue dashed line marks the interface between the two systems. That the environment is well understood is rather crucial, and often leads to the environment itself being tightly constrained. One does not need to know the state of the environment, but it is necessary to know what states are available, and to be able to precisely define its interaction with the system of interest.}
\label{fig:oqs1-1}
\end{figure}


\section{Born, Markov, Secular}

\section{Dissipation, Decoherence, Dephasing}

\section{The Liouvillian}