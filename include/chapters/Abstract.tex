In this thesis we consider stationary states of dissipative many-body quantum systems. We do so using matrix product operator representations of the system. One can find the stationary state by simulating the time evolution of the system \cite{Vidal2004,Schollwock2011}, or by using a more recently proposed variational search technique \cite{Cui2015,Mascarenhas2015}. An implementation of the variational search technique was written for MATLAB \cite{otb:gitVSSS,MATLAB}. Documentation is included in \cref{chp:mpostat}.

In ``Dissipation-induced mobility and coherence in frustrated lattices'' we considered a geometrically frustrated lattice system, in which particles cannot move coherently \cite{Owen2017}. We found that local Markovian dissipation can induce mobility and long-range first-order coherence in the system. This was true in both the non-interacting and interacting regime, though strong interactions suppress the effect.

In ``Localization to delocalization crossover in a driven nonlinear cavity array'' we considered an array of nonlinear cavities, with a coherent parametric drive to the doubly excited state \cite{Brown2018}. The dissipation rate on each site increases with the excitation number. We found that when the hopping rate between sites is low the system forms an incompressible state with commensurate filling, analogous to the Mott insulator. When the hopping rate increases there is a crossover to a delocalized state. In contrast to the equilibrium case, long-range correlations do not build up.

We conclude the thesis by considering some initial results from a new investigation, and by commenting on possible future directions for the variational stationary state search code.