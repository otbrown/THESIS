In this chapter we present some initial work on a promising project begun right at the very end of my PhD, which it is hoped the group will eventually bring to fruition. We will also consider future applications of the variational stationary state search code.

\section{Biased chain}
In this project we will consider an \(N\)-site biased spin chain, governed by the Hamiltonian,
\begin{equation}
	\mathcal{H} = \sum_{j}^{N}\left[ -j \Delta E \hat{a}_{j}^{\dagger}\hat{a}_{j} + t\left( \hat{a}_{j}^{\dagger}\hat{a}_{j+1} + \hat{a}_{j}\hat{a}_{j+1}^{\dagger}\right)\right],
	\label{eq:fw1-1}
\end{equation}
where \(\Delta E\) is the bias energy, and \(t\) is the hopping rate between sites. The Lindblad form master equation for the system is, 
\begin{align}
	\dot{\rho} = -i[\mathcal{H}, \rho] &+ \frac{\gamma_{d}}{2} \sum_{j} \left[2\kappa_{j}\rho\kappa_{j}^{\dagger} - \{\kappa_{j}^{\dagger}\kappa_{j}, \rho\}\right] \notag \\
	&+ \frac{\gamma_{P}}{2}\left[2\hat{a}_{1}^{\dagger}\rho\hat{a}_{1} - \{\hat{a}_{1}\hat{a}_{1}^{\dagger}, \rho\}\right] \notag \\
	&+ \frac{\gamma_{D}}{2}\left[2\hat{a}_{N}\rho\hat{a}_{N}^{\dagger} - \{\hat{a}_{N}^{\dagger}\hat{a}_{N}, \rho\}\right],
	\label{eq:fw1-2}
\end{align}
where \(\gamma_{d}\) is the dissipation rate, \(\gamma_{P}\) is the pump rate of an incoherent pump on the first site, and \(\gamma_{D}\) is the drain rate on the last site. The system (excluding the dissipator) is sketched in \cref{fig:fw1-1}.

We have neglected to explain the dissipative operator \(\kappa_{j}\) so far, as the dissipative environment is complicated. We discuss it in detail next.

\begin{figure}[ht!]
	\centering
	\includegraphics[width=0.95\linewidth]{\figpath/BiasedChain}
	\caption{\label{fig:fw1-1}A sketch of a seven site system. The Hamiltonian for the system is given in \cref{eq:fw1-1}, and the sketch shows the bias energy \(\Delta E\), the hopping interaction with rate \(t\), and the incoherent pump and drain terms with rates \(\gamma_{P}\) and \(\gamma_{D}\). For simplicity, the dissipative interactions are not shown.}
\end{figure}

\subsection{Dissipative environment}
Each of the sites in this system is coupled to an independent dissipative environment. Each environment is approximated as a set of harmonic oscillators,
\begin{equation}
	\mathcal{H}_{E} = \sum_{j,k} \omega\hat{b}_{j,k}^{\dagger}\hat{b}_{j,k},
	\label{eq:fw1-3}
\end{equation}
where the operator \(\hat{b}_{j,k}\) \((\hat{b}_{j,k}^{\dagger})\) annihilates (creates) an excitation in mode \(k\) of the reservoir connected to site \(j\) of the system of the biased spin chain. Each reservoir is then coupled to each site by the system-environment Hamiltonian,
\begin{equation}
	\mathcal{H}_{S\text{-}E} = \sum_{j,k} \left[ g_{j,k}\hat{a}_{j}^{\dagger}\hat{a}_{j}\left( \hat{b}_{j,k} + \hat{b}_{j,k}^{\dagger} \right)\right],
	\label{eq:fw1-4}
\end{equation}
where \(g_{j,k}\) is the coupling frequency. Such an environment leads to Lindblad operators of the form,
\begin{equation}
	\kappa_{j} = \sum_{m,n} \alpha_{m,n}^{j} \hat{a}_{m}^{\dagger}\hat{a}_{n},
	\label{eq:fw1-5}
\end{equation}
where the action of the dissipator is to transfer an excitation from site \(n\) to site \(m\) at a rate determined both by the dissipation rate \(\gamma_{d}\), and the coefficient \(\alpha_{m,n}^{j}\). The coefficients are given by,
\begin{equation}
	\alpha_{m,n}^{j} = \sum_{m,n} \sum_{\varepsilon - \varepsilon' = \omega} \left[ \langle \varepsilon | \hat{a}_{j}^{\dagger}\hat{a}_{j} | \varepsilon' \rangle \langle m |\varepsilon \rangle \langle \varepsilon' | m \rangle \right], 
	\label{eq:fw1-6}
\end{equation}
where \(|\varepsilon \rangle\) is some state from the energy eigenbasis of the system Hamiltonian,
\begin{equation}
	\mathcal{H}|\varepsilon \rangle = \varepsilon|\varepsilon\rangle,
	\label{eq:fw1-7}
\end{equation}
and the state \(|n\rangle\) is not a number state, but here represents the single-excitation state \(|n \rangle = | 0 \rangle_{1} \otimes \ldots \otimes | 1 \rangle_{n} \otimes \ldots \otimes |0 \rangle_{N}\). 

This unusual dissipative environment, including non-local transitions, makes the model both more interesting and more challenging. 

\subsection{Initial results}

\subsection{Next steps}

\section{Future applications} 