In this chapter we present some initial work on a promising project begun right at the very end of my PhD, which it is hoped the group will eventually bring to fruition. We will also consider future applications of the variational stationary state search code, and conclude the thesis.

\section{Biased chain}
In this project we will consider an \(N\)-site biased spin chain, governed by the full system-environment Hamiltonian,
\begin{equation}
	\mathcal{H} = \mathcal{H}_{S} + \mathcal{H}_{S\text{-}E} + \mathcal{H}_{E},
	\label{eq:fw1-0}
\end{equation}
where the system Hamiltonian
\begin{equation}
	\mathcal{H}_{S} = \sum_{j}^{N}\left[ -j \Delta E \hat{a}_{j}^{\dagger}\hat{a}_{j} + t\left( \hat{a}_{j}^{\dagger}\hat{a}_{j+1} + \hat{a}_{j}\hat{a}_{j+1}^{\dagger}\right)\right],
	\label{eq:fw1-1}
\end{equation}
where \(\Delta E\) is the bias energy, and \(t\) is the hopping rate between sites. Each of the sites in this system is coupled to an independent environment. Each environment is approximated as a set of harmonic oscillators,
\begin{equation}
	\mathcal{H}_{E} = \sum_{j,k} \omega\hat{b}_{j,k}^{\dagger}\hat{b}_{j,k},
	\label{eq:fw1-3}
\end{equation}
where the operator \(\hat{b}_{j,k}\) \((\hat{b}_{j,k}^{\dagger})\) annihilates (creates) an excitation in mode \(k\) of the reservoir connected to site \(j\) of the system, and \(\omega\) is the harmonic energy in units of frequency (\(\hbar = 1\)). The system-environment coupling is a dephasing interaction of the form,
\begin{equation}
	\mathcal{H}_{S\text{-}E} = \sum_{j,k} \left[ g_{j,k}\hat{a}_{j}^{\dagger}\hat{a}_{j}\left( \hat{b}_{j,k} + \hat{b}_{j,k}^{\dagger} \right)\right],
	\label{eq:fw1-4}
\end{equation}
where \(g_{j,k}\) is the coupling frequency. Following the usual procedure for deriving a Lindblad-form master equation, including application of the Born-Markov approximation, such an environment leads to relaxation of energy eigenstates of the system Hamiltonian where the energy difference matches the harmonic energy of the environment \(\varepsilon - \varepsilon' = \omega\). In the site basis the Lindblad operators are of the form,
\begin{equation}
	\kappa_{j} = \sum_{m,n} \alpha_{m,n}^{j} \hat{a}_{m}^{\dagger}\hat{a}_{n},
	\label{eq:fw1-5}
\end{equation}
where the action of the dissipator is to transfer an excitation from site \(n\) to site \(m\) at a rate determined both by the dissipation rate \(\gamma_{d}\), and the coefficient \(\alpha_{m,n}^{j}\). The coefficients are given by the projection of energy eigenstates of the system on to the site basis,
\begin{equation}
	\alpha_{m,n}^{j} = \sum_{m,n} \sum_{\varepsilon - \varepsilon' = \omega} \left[ \langle \varepsilon | \hat{a}_{j}^{\dagger}\hat{a}_{j} | \varepsilon' \rangle \langle m |\varepsilon \rangle \langle \varepsilon' | n \rangle \right], 
	\label{eq:fw1-6}
\end{equation}
where \(|\varepsilon \rangle\) is some state from the energy eigenbasis,
\begin{equation}
	\mathcal{H}|\varepsilon \rangle = \varepsilon|\varepsilon\rangle,
	\label{eq:fw1-7}
\end{equation}
and the state \(|n\rangle\) is not a number state, but here represents the single-excitation state \(|n \rangle = | 0 \rangle_{1} \otimes \ldots \otimes | 1 \rangle_{n} \otimes \ldots \otimes |0 \rangle_{N}\). 

The Lindblad form master equation for the system is, 
\begin{align}
	\dot{\rho} = -i[\mathcal{H}, \rho] &+ \frac{\gamma_{d}}{2} \sum_{j} \left[2\kappa_{j}\rho\kappa_{j}^{\dagger} - \{\kappa_{j}^{\dagger}\kappa_{j}, \rho\}\right] \notag \\
	&+ \frac{\gamma_{P}}{2}\left[2\hat{a}_{1}^{\dagger}\rho\hat{a}_{1} - \{\hat{a}_{1}\hat{a}_{1}^{\dagger}, \rho\}\right] \notag \\
	&+ \frac{\gamma_{D}}{2}\left[2\hat{a}_{N}\rho\hat{a}_{N}^{\dagger} - \{\hat{a}_{N}^{\dagger}\hat{a}_{N}, \rho\}\right],
	\label{eq:fw1-2}
\end{align}
where \(\gamma_{d}\) is the dissipation rate, \(\gamma_{P}\) is the pump rate of an incoherent pump on the first site, and \(\gamma_{D}\) is the drain rate on the last site. The system (excluding the dissipator) is sketched in \cref{fig:fw1-1}.

\begin{figure}[ht!]
	\centering
	\includegraphics[width=0.95\linewidth]{\figpath/BiasedChain}
	\caption{\label{fig:fw1-1}A sketch of a seven site system. The Hamiltonian for the system is given in \cref{eq:fw1-1}, and the sketch shows the bias energy \(\Delta E\), the hopping interaction with rate \(t\), and the incoherent pump and drain terms with rates \(\gamma_{P}\) and \(\gamma_{D}\). For simplicity, the dissipative interactions are not shown.}
\end{figure}

\subsection{Initial results}
The dissipator given in \cref{eq:fw1-5,eq:fw1-6} extends across the whole lattice, however this is computationally impractical. As such, in order to solve the system, we apply a cutoff to the \(\alpha_{m,n}\) coefficients, \(\alpha_{m,n} < 0.01\) is zero. Note that since these coefficients are calculated in part using the eigenvectors of the system Hamiltonian \(\mathcal{H}\), they also rely on the system parameters, \(\Delta E\) and \(t\). In particular, the larger the hopping rate \(t\), the longer the range of non-zero alpha coefficents. \Cref{fig:fw1-2} shows the coefficients for the middle site of an eleven site system, \(\alpha_{6,n}^{6}\) against the extent, \(|6-n|\), for a range of values of \( t / \Delta E\). In order to limit the extent of interactions to two sites for initial calculations, we consider only values of the hopping rate and bias energy \(t / \Delta E \leq 0.1\). 

\begin{figure}[ht!]
	\centering
	\includegraphics[width=0.6\linewidth]{\figpath/DissExtent}
	\caption{\label{fig:fw1-2}Plot of the non-local dissipation coefficients, given by \cref{eq:fw1-6}, for the middle site of an eleven site system. The coefficients are plotted against the extent \(|6-n|\). The red dashed line marks the cutoff at \(\alpha_{m,n} = 0.01\).}
\end{figure}

With these parameter restrictions we calculated the steady state of an eleven site system, using the variational stationary state search code. Design of the MPO followed a similar process to that followed for the MPO in Ref.~\cite{Owen2017}, which is discussed in \cref{chp:dimmpo}. For these calculations the energy bias \(\Delta E = 1\), the pump rate \(\gamma_{P} = 0.05\), the drain rate \(\gamma_{D} = 0.05\), the dissipation rate \(\gamma_{d} = 0.05\), and the hopping rate \(t\) is scanned between 0.01 and 0.1. Naturally the results of these calculations are still under investigation, but the plots in \cref{fig:fw1-3} show some features of interest. In particular, the density profile is consistent despite the change in the hopping rate, and that coherent transport undergoes an apparent phase change as hopping rate increases. The dashed line in \cref{fig:fw1-3b} demonstrates this for coherent transport through the bond between the fifth and sixth site in the system.
 
\begin{figure}[ht!]
	\subfloat[\label{fig:fw1-3a}]{\includegraphics[width=0.49\linewidth]{\figpath/BC_AbsN}} \hfill
	\subfloat[\label{fig:fw1-3b}]{\includegraphics[width=0.49\linewidth]{\figpath/BC_CohTrans}}
	\caption{\label{fig:fw1-3}(a) The density of an eleven site system \(\langle \hat{n}_{j} \rangle\), against the site \(j\), and hopping rate \(t\). (b) Coherent transport between sites in the lattice, against site \(j\), and hopping rate \(t\). In both these calculations the bias energy \(\Delta E = 1\), and the pump, drain, and dissipation rates \(\gamma_{P} = \gamma_{D} = \gamma_{d} = 0.05\).}
\end{figure}

\subsection{Next steps}
Most immediately we would like to carry out other calculations to verify that the results we have are accurate. The variational search method cannot guarantee physical results, but confidence can be improved by running calculations with lower convergence thresholds (and higher matrix dimensions). One possibility would be to extend the range of the MPO, which would allow us to either reduce the alpha cutoff, or increase the hopping rate. We would also like to explicitly calculate the incoherent transport in the system, and to consider different configurations of the system. 

\section{Conclusions} 
In this thesis we began by introducing the scaling problem, the fundamental difficulty of many-body quantum physics, and then went on to describe in some detail the driven Bose-Hubbard model which the research presented focuses on. In addition we provide some physical motivation for such a model, in an attempt to avoid criticism for being too abstracted from reality. Next we introduced the standard tools of open quantum systems -- the Lindblad-form master equation, and the Liouvillian matrix. Ultimately the problem of finding the stationary state of a driven-dissipative system is the same as finding the ground state of a closed system, except that the matrix is much larger! In the next we chapter we considered one method for overcoming the challenge this presents computationally -- matrix product states (and operators). We go in to a lot of detail on one particular technique, the variational search, and for good reason. The \lstinline$mpostat$ code found in the online repository hosted at \cite{otb:gitVSSS}, and available under an open source license, implements this technique. It is one of the principle outputs of my PhD, and the documentation is included in this thesis in \cref{chp:mpostat}. Like any technique, the variational search has both advantages and disadvantages. It is generally quicker than time-evolution methods, and it is easy to encode long range interactions in the matrix product operator format used to specify the system to be solved. On the other hand, it is memory hungry, and the result is not guaranteed to be physical, so it must be used with care. Nevertheless, as is shown in ref.~\cite{Owen2017}, in the work presented in this chapter, and in work done by others \cite{Cui2015,Mascarenhas2015}, the technique is powerful enough to push at and expand the boundaries of what is possible in this challenging field. We then considered the two research papers produced during my PhD, ref.~\cite{Owen2017} (published) and ref.~\cite{Brown2018} (submitted), and expanded on what is available in their original format. Finally, in this chapter we have discussed one of the new projects that we have begun working on, and which will make use of the variational search code.

As for the code itself, future directions for improvement would include making it more flexible, while not compromising its focus. Other libraries such as the TNT Library \cite{TNTlib,Al-Assam2017} implement matrix product state methods more generally, and do it well, and it would be unwise to attempt to replicate that effort by allowing \lstinline$mpostat$ to bloat. One specific recommendation would be to simplify the data structures by reimplementing matrix product states and operators as classes rather than cell arrays, which could make it easier for users to write the MPO for any given system, and would make it easier to make the code more flexible. If I were to have another four years to work on the code, that is one of the first things I would do! Other than that, I would continue to make improvements to the top-level control logic and reporting, so as to make the code more user-friendly, and I would introduce some automatic testing of the results. 

It is my hope that \lstinline$mpostat$ will continue to used to investigate challenging systems in the field of driven dissipative many-body quantum systems, and will continue to be improved, and I am pleased that it has already proven useful. That is all, thank you for reading.