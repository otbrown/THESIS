In this chapter we present some initial work on a promising project begun right at the very end of my PhD, which it is hoped the group will eventually bring to fruition.

\section{Biased chain}
In recent work a biased spin chain, where the excited state on each site has a lower energy level than the excited state on all preceding sites, was used as a model to describe the quantum dynamics of a photocell \cite{Fruchtman2016}. This investigation however only considered the subspace in which there is a single excitation in the system. We are interested in considering a similar system in which many excitations exist. In particular, we are interested in the transport properties of such a system -- in the previous work the system was considered as a quantum heat engine where work is extracted through the drain on the last site \cite{Dorfman2013}. Consequently the question of how to maximise the current from pump to drain is an interesting one. We are also interested in it more generally as a system which may exhibit incoherent transport, as we were in reference~\cite{Owen2017}. The system (excluding the dissipator) is sketched in \cref{fig:fw1-1}.

\begin{figure}[ht!]
	\centering
	\includegraphics[width=0.95\linewidth]{\figpath/BiasedChain}
	\caption{\label{fig:fw1-1}A sketch of a seven site system. The Hamiltonian for the system is given in \cref{eq:fw1-1}, and the sketch shows the bias energy \(\Delta E\), the hopping interaction with rate \(t\), and the incoherent pump and drain terms with rates \(\gamma_{P}\) and \(\gamma_{D}\). For simplicity, the dissipative interactions are not shown.}
\end{figure}

\subsection{The model}
We consider an \(N\)-site spin chain governed by the full system-environment Hamiltonian,
\begin{equation}
	\mathcal{H} = \mathcal{H}_{S} + \mathcal{H}_{S\text{-}E} + \mathcal{H}_{E},
	\label{eq:fw1-0}
\end{equation}
where the system Hamiltonian
\begin{equation}
	\mathcal{H}_{S} = \sum_{j}^{N}\left[ -j \Delta E \hat{a}_{j}^{\dagger}\hat{a}_{j} + t\left( \hat{a}_{j}^{\dagger}\hat{a}_{j+1} + \hat{a}_{j}\hat{a}_{j+1}^{\dagger}\right)\right],
	\label{eq:fw1-1}
\end{equation}
where \(\Delta E\) is the bias energy, and \(t\) is the hopping rate between sites. Each of the sites in this system is coupled to an independent environment. Each environment is approximated as a set of harmonic oscillators,
\begin{equation}
	\mathcal{H}_{E} = \sum_{j,k} \omega\hat{b}_{j,k}^{\dagger}\hat{b}_{j,k},
	\label{eq:fw1-3}
\end{equation}
where the operator \(\hat{b}_{j,k}\) \((\hat{b}_{j,k}^{\dagger})\) annihilates (creates) an excitation in mode \(k\) of the reservoir connected to site \(j\) of the system, and \(\omega\) is the harmonic energy in units of frequency (\(\hbar = 1\)). The system-environment coupling is a dephasing interaction of the form,
\begin{equation}
	\mathcal{H}_{S\text{-}E} = \sum_{j,k} \left[ g_{j,k}\hat{a}_{j}^{\dagger}\hat{a}_{j}\left( \hat{b}_{j,k} + \hat{b}_{j,k}^{\dagger} \right)\right],
	\label{eq:fw1-4}
\end{equation}
where \(g_{j,k}\) is the coupling frequency. Following the usual procedure for deriving a Lindblad-form master equation, including application of the Born-Markov approximation, such an environment leads to relaxation of energy eigenstates of the system Hamiltonian where the energy difference matches the harmonic energy of the environment \(\varepsilon - \varepsilon' = \omega\) \cite{Beaudoin2011}. In the site basis the Lindblad operators are of the form,
\begin{equation}
	\kappa_{j} = \sum_{m,n} \alpha_{m,n}^{j} \hat{a}_{m}^{\dagger}\hat{a}_{n},
	\label{eq:fw1-5}
\end{equation}
where the action of the dissipator is to transfer an excitation from site \(n\) to site \(m\) at a rate determined both by the dissipation rate \(\gamma_{d}\), and the coefficient \(\alpha_{m,n}^{j}\). The coefficients are given by the projection of energy eigenstates of the system on to the site basis,
\begin{equation}
	\alpha_{m,n}^{j} = \sum_{m,n} \sum_{\varepsilon - \varepsilon' = \omega} \left[ \langle \varepsilon | \hat{a}_{j}^{\dagger}\hat{a}_{j} | \varepsilon' \rangle \langle m |\varepsilon \rangle \langle \varepsilon' | n \rangle \right],
	\label{eq:fw1-6}
\end{equation}
where \(|\varepsilon \rangle\) is some state from the energy eigenbasis,
\begin{equation}
	\mathcal{H}|\varepsilon \rangle = \varepsilon|\varepsilon\rangle,
	\label{eq:fw1-7}
\end{equation}
and the state \(|n\rangle\) is not a number state, but here represents the single-excitation state \(|n \rangle = | 0 \rangle_{1} \otimes \ldots \otimes | 1 \rangle_{n} \otimes \ldots \otimes |0 \rangle_{N}\).

The Lindblad form master equation for the system is,
\begin{align}
	\dot{\rho} = -i[\mathcal{H}, \rho] &+ \frac{\gamma_{d}}{2} \sum_{j} \left[2\kappa_{j}\rho\kappa_{j}^{\dagger} - \{\kappa_{j}^{\dagger}\kappa_{j}, \rho\}\right] \notag \\
	&+ \frac{\gamma_{P}}{2}\left[2\hat{a}_{1}^{\dagger}\rho\hat{a}_{1} - \{\hat{a}_{1}\hat{a}_{1}^{\dagger}, \rho\}\right] \notag \\
	&+ \frac{\gamma_{D}}{2}\left[2\hat{a}_{N}\rho\hat{a}_{N}^{\dagger} - \{\hat{a}_{N}^{\dagger}\hat{a}_{N}, \rho\}\right],
	\label{eq:fw1-2}
\end{align}
where \(\gamma_{d}\) is the dissipation rate, \(\gamma_{P}\) is the pump rate of an incoherent pump on the first site, and \(\gamma_{D}\) is the drain rate on the last site.

\subsection{Initial results}
The dissipator given in \cref{eq:fw1-5,eq:fw1-6} extends across the whole lattice, however this is computationally impractical. As such, in order to solve the system, we apply a cutoff to the \(\alpha_{m,n}\) coefficients, so that \(\alpha_{m,n} < 0.01\) is set to zero. Note that since these coefficients are calculated in part using the eigenvectors of the system Hamiltonian \(\mathcal{H}\), they also rely on the system parameters, \(\Delta E\) and \(t\). In particular, the larger the hopping rate \(t\), the longer the range of non-zero alpha coefficients. \Cref{fig:fw1-2} shows the coefficients for the middle site of an eleven site system, \(\alpha_{6,n}^{6}\) against the extent, \(|6-n|\), for a range of values of \( t / \Delta E\). In order to limit the extent of interactions to two sites for initial calculations, we consider only values of the hopping rate and bias energy \(t / \Delta E \leq 0.1\).

\begin{figure}[ht!]
	\centering
	\includegraphics[width=0.6\linewidth]{\figpath/DissExtent}
	\caption{\label{fig:fw1-2}Plot of the non-local dissipation coefficients, given by \cref{eq:fw1-6}, for the middle site of an eleven site system. The coefficients are plotted against the extent \(|6-n|\). The red dashed line marks the cutoff at \(\alpha_{m,n} = 0.01\).}
\end{figure}

With these parameter restrictions we calculated the steady state of an eleven site system, using the variational stationary state search code. Design of the MPO followed a similar process to that followed for the MPO in reference~\cite{Owen2017}, which is discussed in \cref{chp:dimmpo}. For these calculations the energy bias \(\Delta E = 1\), the pump rate \(\gamma_{P} = 0.05\), the drain rate \(\gamma_{D} = 0.05\), the dissipation rate \(\gamma_{d} = 0.05\), and the hopping rate \(t\) is scanned between 0.01 and 0.1. Naturally the results of these calculations are still under investigation, but the plots in \cref{fig:fw1-3,fig:fw1-4} show some features of interest. In particular, the density profile in \cref{fig:fw1-3} remains the same despite the change in the hopping rate. In \cref{fig:fw1-4} it can be seen that coherent transport, given by
\begin{equation}
	i\langle [\mathcal{H}_{S}, \hat{a}_{j}^{\dagger}\hat{a}_{j}] \rangle =  \langle it \left(\hat{a}_{j}^{\dagger}\hat{a}_{j+1} - \hat{a}_{j}\hat{a}_{j+1}^{\dagger}\right) \rangle,
	\label{eq:fw1-8}
\end{equation}
undergoes an apparent phase change as the hopping rate increases. The dashed line in \cref{fig:fw1-4a} demonstrates this for coherent transport through the bond between the fifth and sixth sites in the system -- the same line is plotted separately in \cref{fig:fw1-4b}.

\begin{figure}[ht!]
	\centering
	\includegraphics[width=0.6\linewidth]{\figpath/BC_AbsN}
	\caption{\label{fig:fw1-3}The density of an eleven site system \(\langle \hat{n}_{j} \rangle\), against the site \(j\), and hopping rate \(t\). In this calculation the bias energy \(\Delta E = 1\), and the pump, drain, and dissipation rates \(\gamma_{P} = \gamma_{D} = \gamma_{d} = 0.05\).}
\end{figure}

\begin{figure}[ht!]
	\subfloat[\label{fig:fw1-4a}]{\includegraphics[width=0.49\linewidth]{\figpath/BC_CohTrans}} \hfill
	\subfloat[\label{fig:fw1-4b}]{\includegraphics[width=0.49\linewidth]{\figpath/BC_CohTransHop}}
	\caption{\label{fig:fw1-4}(a) Coherent transport between sites in the lattice, against site \(j\), and hopping rate \(t\). (b) Coherent transport between sites five and six in the lattice, against hopping rate \(t\). In this calculation the bias energy \(\Delta E = 1\), and the pump, drain, and dissipation rates \(\gamma_{P} = \gamma_{D} = \gamma_{d} = 0.05\).}
\end{figure}

\subsection{Next steps}
Most immediately we would like to carry out other calculations to verify that the results we have are accurate. The variational search method cannot guarantee physical results, but confidence can be improved by running calculations with lower convergence thresholds (and higher matrix dimensions).

From the initial results we anticipate that the apparent phase transition from incoherent to coherent transport evidenced in \cref{fig:fw1-4b} may become the focus of our investigation, rather than how transport might be maximised as we initially expected. For both lines of investigation we would like to calculate the full derivative of the density on some site \(j\), rather than just the contribution from the coherent dynamics. This would particularly help us understand what is happening in the region where coherent transport is lower.

We would also like to extend the range of the MPO, which would allow us to either reduce the alpha cutoff, or increase the hopping rate, and to consider different configurations of the system.
